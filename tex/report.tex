\documentclass[a4paper,12pt,pagesize,DIV=calc,smallheadings,english]{article}

%% Packages
% Document margins/formating
\usepackage[margin=2.5cm]{geometry}
\usepackage{setspace} 
\setstretch{1.5} % Zeilenabstand definieren
% Language and fonts
\usepackage[utf8]{inputenc} % Allows Umlauts
\usepackage[osf,sc]{mathpazo} % Use the Palatino font
\usepackage[T1]{fontenc} % western enc. Improves wordsplitting
\usepackage[english]{babel} % Set language
\usepackage[tracking=true]{microtype} % Slightly tweak font spacing for aesthetics
\DeclareMicrotypeSet*[tracking]{my}% Anleitung lesen um das zu verstehen.
  { font = */*/*/sc/* }%
\SetTracking{ encoding = *, shape = sc }{ 45 }%
\usepackage{ellipsis} % Slightly tweak spacing for \dots
% Math fonts and symbols
\usepackage{mathtools}
\usepackage{bm} % For bolding mathsymbols and letters
\usepackage{amssymb}
% Additional formating (optional)
\usepackage[bf]{titlesec}
\usepackage[margin=15pt,hang,small,labelfont={bf,sf},up,up]{caption} % Custom captions 
% Bibliography
\usepackage[backend=biber,style=authoryear,sorting=nyt]{biblatex}
\usepackage[autostyle,autopunct]{csquotes} % For Quotations
\renewcommand{\mkcitation}[1]{#1} % For correct footcites with csquotes
\renewcommand*{\mkbibnamefamily}[1]{\textsc{#1}} % For small caps author names
\addbibresource{references.bib} % Name of Biblatex file
% Table formating
\usepackage{array}
\usepackage{multirow}
\usepackage{booktabs}
\setlength{\heavyrulewidth}{1.5pt}
\setlength{\abovetopsep}{4pt}
% Additional packages
\usepackage[hidelinks]{hyperref} % For hyperlinks in the PDF
\usepackage{tikz}
\usepackage{import}
\usepackage{enumitem} 
\usepackage{verbatim}
\usepackage{siunitx}
\usepackage{graphicx}
\usepackage{subcaption}
\usepackage{color}
\usepackage{transparent}
\usepackage{comment}
\usepackage{tcolorbox}
\usepackage{amsthm}
\usepackage{mdframed}

% Shortcuts/Definitions
\DeclareMathOperator*{\argmin}{argmin}   % Jan Hlavacek
\DeclareMathOperator*{\argmax}{argmax}   % Jan Hlavacek
\DeclareMathOperator*{\arginf}{arginf}   % Jan Hlavacek

% Theorems and Definitions
\newtheorem{theorem}{Theorem}[section]
\newtheorem{corollary}{Corollary}[theorem]
\newtheorem{lemma}[theorem]{Lemma}
\newtheorem{remark}[theorem]{Remark}
\newtheorem{definition}[theorem]{Definition}

% Long mapsto arrows
%\usepackage{stmaryrd}

% Extra settings
\allowdisplaybreaks



%% The document starts here!
\begin{document}


% --- Front Matter

\begin{titlepage}

  \vspace*{0.5cm}

  \begin{center}
    {\LARGE\textbf{Elastic Full Procrustes Means\vspace{0.4cm}\\
    for Sparse and Irregular Planar Curves}}
  \vspace{1cm}\\
    {\large Masters Thesis}
  \vspace{0.7cm}\\
  \textsc{Manuel Pfeuffer\footnote{\url{pfeuferm@hu-berlin.de}, Matriculation Number: 577668}}
  \vspace{0.1cm}\\
  12th October 2020, Berlin
  \vspace{1cm}\\

  \vfill
  \end{center}

  \noindent \textbf{Advisors:} Lisa Maike Steyer, Almond Stöcker\\
  \noindent \textbf{1st Examiner:} Prof.\ Dr.\ Sonja Greven\\
  \noindent \textbf{2nd Examiner:} Prof.\ Dr.\ Nadja Klein
  \vspace{0.5em}

\end{titlepage}

\pagenumbering{roman}
\tableofcontents


% --- Main Content

\newpage
\pagenumbering{arabic}
\section{Introduction}
\label{sec:intro}
In statistical shape analysis one is interested in analysing the shape of
geometrical objects, like the outlines of bones and organs, a handwritten
digit, or the folds of a protein.
When analysing such objects, differences in location, rotation, and size
are often not of interest.
Instead, the focus lies purely on their differences in  \textit{shape}, a
widely adapted definition of which was established by \cite{Kendall1977} and which
might be formulated in the following way:
\begin{definition}[Shape] 
    All geometrical information that remains when location, scale
    and rotational effects are removed from an object 
    \parencite[see][1]{DrydenMardia2016}.
\end{definition}
\noindent This geometrical information is usually approximated by measuring
the cordinates of a fixed set of \textit{landmarks}, which are characteristic
points on an object that match between and within populations
\parencite[see][3]{DrydenMardia2016}.
As these coordinates depend on the scale, rotation and translation of the
object at time of measurement, only the \textit{equivalence class} of the
landmark configuration modulo these transformations is indicative of its shape,
which will be defined in more detail in section \ref{theo:inv}.
A popular type of shape mean that does not depend on the rotation, scaling and
translation of the input shapes is the \textit{full Procrustes mean}.
Here, the mean is defined as the minimizer of a least squares criterion, using
a distance measure that is invariant under the mentioned transformations, which
will be discussed in section \ref{theo:proc}.

Instead of approximating the geometry of an object using landmarks, its whole
outline might be represented by an open or closed curve $\beta : [0,1]
\rightarrow \mathbb{R}^d,\, d \in \mathbb{N}$, eliminating the (often
subjective) decision of which points to consider as "characteristic".
This functional data approach introduces an additional kind of shape invariance
relating to the parametrisation $t \in [0,1]$ of $\beta$, as only the image of
$\beta$ but not its parametrisation is indacative of the objects shape.
A functional shape mean, which is invariant with respect to re-parametrisation
of the input curves, is called an \textit{elastic} mean and its calculation is
greatly simplified by working in the \textit{square-root-velocity} (SRV)
framework as introduced in \cite{SrivastavaEtAl2011}.
Elastic shape means and the SRV framework will be explained in section
\ref{theo:srv}.
Even though this functional approach eliminates the subjectivity of choosing a
fixed set of landmarks, in practice a curve will never be fully observed and
one has to again work with a set of sampled points along the curve. 
A particular challenge is working with such curves, that are sparsely and
irregularly sampled, where one has to employ appropriate smoothing techniques
to make maximum use of the available (sparse) data.
This will be briefly discussed in section \ref{theo:sparse}.

The aim of this thesis is to bring all these concepts together in the
estimation of \enquote{Elastic Full Procrustes Means for Sparse and Irregular
Planar Curves}. 
The thesis focuses on the special case of 2D (i.e. planar) curves, as it can be
shown that the Procrustes mean as particularly nice properties in this
settings.
Additionally this thesis will be mainly concerned with the mean estimation for
open curves, as the closed curves case is more challenging mathematically.
After covering the relevant background material in sufficent detail in section
\ref{sec:theo}, an expression for the elastic full Procrustes mean will be
derived in section \ref{sec:mean}.
In section \ref{sec:algo}, an estimation strategy for the setting of sparse and
irregular curves is proposed, which will be applied to simulated and empirical
datasets in section \ref{sec:app}.
Finally, in section \ref{sec:outl} a possible extension to closed curves will
be briefly discussed.
All results will be briefly summarized in section \ref{sec:sum}.

\newpage


\newpage
\section{Functional and Shape Data Analysis of Planar Curves}
\label{sec:theo}
Before beginning with a derivation of the elastic full Procrustes mean, it is
important to establish a notational and mathematical framework for the
treatment of planar shapes.
While the restriction to the 2D case might seem a major one, it still covers
all shape data extracted from e.g.\ imagery and is therefore very applicable in
practice.
Examples of objects that can be analyzed in this way are handwritten digits or
symbols, the outlines bones and organs in medical images, or even movement
trajectories on a map.

All these objects may be naturally represented as planar curves $\beta : [0,1]
\rightarrow \mathbb{R}^2$ with $\beta(t) = (x(t),\, y(t))^T$, where $x(t)$ and
$y(t)$ are the scalar-valud \textit{coordinate functions}.
Calculations in 2D, and in particular the derivation of the
full Procrustes mean, are greatly simplified by using complex notation.
Going forward, we will therefore identify $\mathbb{R}^2$ with $\mathbb{C}$ and
always use complex notation when representing a planar curve:
$$\beta : [0,1] \rightarrow \mathbb{C}, \quad \beta(t) = x(t) + i\, y(t).$$

\subsection{Shape Invariance and Equivalence Classes}
\label{theo:inv}
The concept of shape is closely related to the concept of invariance under the
transformations of scaling, translation and rotation.
When considering the shape of a curve, we additionally have to take into
account invariance with respect to re-parametrisation.
This can be seen by noting that the curves $\beta(t)$ and $\beta(\gamma(t))$,
with some \textit{warping function} $\gamma : [0,1] \rightarrow [0,1]$
monotonically increasing and differentiable, have the same image and therefore
represent the same geometrical object.
We can say that the actions of translation, scaling, rotation and
re-parametrization are \textit{equivalence relations} with respect to shape, as
each action leavs the shape of the curve untouched and only changes the way the
shape is represented.
The shape of a curve can then be defined as the respective \textit{equivalence
class}, i.e. the set of all possible shape preserving transformations of the
curve.
As two equivalence classses are neccessarily either disjoint or identical, we
can consider two curves as having the same shape, if they are elements of the
same equivalence class \parencite[see][40]{SrivastavaKlassen2016}.

\begin{definition}[Equivalence relation, equivalence class and quotient space] 
    A relation $\backsim$ on a set $X$ is called an \textbf{equivalence
    relation} if, for all $x,y,z \in X$, it has the following properties:
    \begin{itemize}[noitemsep,topsep=0pt]
        \item[i.] $x \backsim x$ (reflexivity)
        \item[ii.] $x \backsim y \Rightarrow y \backsim x$ (symmetry)
        \item[iii.] $x \backsim y, y \backsim z \Rightarrow x \backsim z$
        (transitivity)
    \end{itemize}
    The \textbf{equivalence class} $[x]$ of $x \in X$ is given by the set of
    all $y \in X$ so that $x \backsim y$.
    The \textbf{quotient space} $X \big/ {\backsim}$ of $X$ under the relation
    $\backsim$ is the disjoint set of all equivalence classes in $X$. 
\end{definition}

We now want to define an equivalence relation with respect to shape.
With this in mind, let us first consider how the discussed transformations act
individually on the set of parametrized planar curves with complex
representation $\beta : [0,1]
\rightarrow \mathbb{C}$:
\begin{itemize}
  \item
    The \textbf{translation} group $\mathbb{C}$ acts on $\beta$ by $(\xi, \beta)
    \mapsto \beta + \xi$, for any $\xi \in \mathbb{C}$.
    We can consider two curves as equivalent with respect to translation
    $\beta_1 \backsim \beta_2$, if there exists a complex scalar $\tilde\xi \in
    \mathbb{C}$ so that $\beta_1 = \beta_2  + \tilde\xi$.
    For this relation an equivalence class is $[\beta] = \{\beta + \xi\, |\,
    \beta : [0,1] \rightarrow \mathbb{C},\, \xi \in \mathbb{C}\}$.
  \item 
    The \textbf{scaling} group $\mathbb{R}^+$ acts on $\beta$ by $(\lambda, \beta)
    \mapsto \lambda \beta$, for any $\lambda \in \mathbb{R}^+$.
    We define $\beta_1 \backsim \beta_2$, if there exists a scalar
    $\tilde\lambda \in \mathbb{R}^+$ so that $\beta_1 = \tilde\lambda \beta_2$.
    An equivalence class is $[\beta] = \{\lambda\beta\,|\,\beta : [0,1]
    \rightarrow \mathbb{C},\, \lambda \in \mathbb{R}^+\}$.
  \item 
    The \textbf{rotation} group $[0,2\pi]$ acts on $\beta$ by $(\theta, \beta)
    \mapsto e^{i\theta} \beta$, for any $\theta \in [0,2\pi]$.
    We define $\beta_1 \backsim \beta_2$, if there exists a $\tilde\theta \in
    [0,2\pi]$ with $\beta_1 = e^{i\tilde\theta} \beta_2$.
    An equivalence class is $[\beta] = \{e^{i\theta}\beta\,|\, \beta : [0,1]
    \rightarrow \mathbb{C},\, \theta \in [0,2\pi]\}$.
  \item 
    The \textbf{re-parametrization} group $\Gamma$ acts on $\beta$ by $(\gamma,\beta)
    \mapsto \beta \circ \gamma$, for any $\gamma \in \Gamma$ with $\Gamma$
    being the set of monotonically increasing and differentiable warping
    functions.
    We define $\beta_1 \backsim \beta_2$, if there exists a warping function
    $\tilde\gamma \in \Gamma$ with $\beta_1 = \beta_2 \circ \tilde\gamma$.
    An equivalence class is $[\beta] = \{\beta \circ \gamma\,|\, \beta : [0,1]
    \rightarrow \mathbb{C},\, \gamma \in \Gamma\}$.
\end{itemize}
In a next step, we can consider how these transformations act in concert and
whether they \textit{commute}, that is, whether the order of applying the
transformations changes outcomes.
Consider for example the actions of the rotation and scaling product group
$\mathbb{R}^+ \times [0,2\pi]$ given by $((\lambda, \theta), \beta) \mapsto
\lambda e^{i\theta} \beta$, which clearly commute as $\lambda
(e^{i\theta}\beta) = e^{i\theta}(\lambda\beta)$.
However, the joint actions of scaling, rotation, and translation do not
commute, as $\lambda e^{i\theta}(\beta + \xi) \neq \lambda e^{i\theta}\beta +
\xi$ and therefore, the order of translating and rotating or scaling matters.
\textit{Euclidean similarity transformations}.
\begin{definition}[Euclidean similarity transformation] 
  We define an \textbf{Euclidean similarity transformation} on a curve $\beta :
  [0,1] \rightarrow \mathbb{C}$ as the joint action of scaling, rotation, and
  translation by $((\xi, \lambda, \theta), \beta) \mapsto \lambda e^{i\theta}
  \beta + \xi$, with $\xi \in \mathbb{C}$, $\lambda \in \mathbb{R}^+$, and
  $\theta \in [0,2\pi]$ \parencite[see][62]{DrydenMardia2016}.
\end{definition}
With respect to the action of re-parametrization, we can note that it
necessarily commutes with all Euclidean similarity transformations as those
only change the image of $\beta$, while the former only changes the
parametrization.
Putting everything together, we can finally give a mathematical definition of
the shape of a planar curve.
\begin{definition}[Shape]
  The \textbf{shape} of a planar curve $\beta : [0,1] \rightarrow \mathbb{C}$
  is given by the equivalence class with respect to all Euclidean similarity
  transformations and re-parametrizations of $\beta$
  $$ [\beta] = \{\lambda e^{i\theta}(\beta \circ \gamma) + \xi\,|\, \xi \in
  \mathbb{C},\, \lambda \in \mathbb{R}^+,\, \theta \in [0,2\pi],\, \gamma \in
  \Gamma\} $$
\end{definition}





\subsection{The Full Procrustes Mean for Planar Curves}
\label{theo:proc}
\begin{definition}[Full Procrustes distance, full Procrustes mean]
    For $X_1, X_2$ landmark configurations, represented as $m \times d$
    matrices with $m$ landmarks in $d$ dimensions, the \textbf{full Procrustes
    distance} between their shapes $[X_1], [X_2]$ is defined as
    $$d_F([X_1], [X_2]) = 
      \inf_{\lambda \in \mathbb{R}_+,\, \Gamma \in SO(m)} ||\widetilde{X}_1 - \lambda
      \widetilde{X}_2\Gamma||, $$
    where $\widetilde{X}_{1,2}$ are centered and normalized landmark
    configurations, $\lambda \in \mathbb{R}_+$ is a scaling factor and $\Gamma
    \in SO(d)$ a rotation matrix.

    The \textbf{full Procrustes mean} shape for a sample of landmark
    configurations $X_i$ ($i = 1,\dots,n$) is then given by the equivalence
    class [$\hat\mu_F$] of a landmark configuration that minimizes the sum of
    squared full Procrustes distances
    $$\hat{\mu}_F = \arginf_{\mu} \sum_{i=1}^n d_F([\mu], [X_i])^2, $$
    where $\mu$ is assumed centered and normalized
    \parencites[see][71,114]{DrydenMardia2016}.
\end{definition}


\subsection{Elastic Means and the Square-Root-Velocity Framework}
\label{theo:srv}


\subsection{Functional Data Analysis of Sparse and Irregular Planar Curves}
\label{theo:sparse}


\newpage


\newpage
\section{The Elastic Full Procrustes Means for Planar Curves}
\label{sec:mean}

Let $\beta$ be a continuous planar curve.
It can be represented in a parameterized form in $\mathbb{R}^2$ as
$$ \beta : [0,1] \rightarrow \mathbb{R}^2,\quad \beta(t) = ( x(t), y(t)) \,, $$
where $x, y$ are scalar-valued \textit{coordinate functions} of $\beta$, parametrized by $t$.
We can equivalently represent a planar curve using complex numbers as
$$ \beta : [0,1] \rightarrow \mathbb{C},\quad \beta(t) = x(t) + iy(t) \,, $$
with the added benefit that complex notation often simplifies calculations in the 2D case.

For a set of planar curves $\beta_1,\dots,\beta_n : [0,1] \rightarrow \mathbb{C}$, either centered with $\langle \beta_i, \mathbb{1} \rangle$ or with no relative translation to each other, the \textit{full Procrustes mean} $\hat{\mu}$ is then defined as the curve minimizing the sum of squared \textit{full Procrustes distances} from each $\beta_i$ to an unknown unit size mean configuration $\mu$, that is
\begin{align*}
    \hat{\mu} =& \argmin_{\mu:[0,1]\rightarrow\mathbb{C}} \sum_{i=1}^n d^2_F(\mu,\beta_i)
    \quad\text{s.t.}\,\, ||\mu|| = 1 \\
    =& \argmin_{\mu:[0,1]\rightarrow\mathbb{C}} \sum_{i=1}^n 1 - \frac{\langle \mu, \beta_i \rangle \langle \beta_i, \mu \rangle}{\langle \mu, \mu \rangle \langle \beta_i, \beta_i \rangle}
    \quad\text{s.t.}\,\, ||\mu|| = 1
\end{align*}
which we can be further simplified by normalizing $\beta_i := \frac{\beta_i}{|| \beta_i ||}$ and using $\langle \mu, \mu \rangle = 1$
$$ \hat{\mu} = \argmax_{\mu:[0,1]\rightarrow\mathbb{C}} \sum_{i=1}^n \langle \mu, \beta_i \rangle \langle \beta_i, \mu \rangle \quad\text{s.t.}\,\, ||\mu|| = 1. $$
The expression for $d^2_F(\mu,\beta_i)$ in the case of planar curves is derived in appendix \ref{app:deriv-full-proc-dist}.


\subsection{The Full Procrustes mean}
Consider a set of planar SRV curves $q_1,\dots,q_n : [0,1] \rightarrow \mathbb{C}$ of unit length $||q_i|| = 1$ for all $i$.
The \textit{full Procrustes mean} $\hat{\mu}$ is given by
\begin{align*}
    \hat{\mu} =& \argmax_{\mu:[0,1]\rightarrow\mathbb{C}} \sum_{i=1}^n \langle \mu, q_i \rangle \langle q_i, \mu \rangle
    \quad\text{s.t.}\,\, ||\mu|| = 1 \\
    =& \argmax_{\mu:[0,1]\rightarrow\mathbb{C}} \sum_{i=1}^n
    \int_0^1 \overline{\mu(t)} q_i(t) \, dt \int_0^1 \overline{q_i(s)} \mu(s) \, ds
    \quad\text{s.t.}\,\, ||\mu|| = 1 \\
    =& \argmax_{\mu:[0,1]\rightarrow\mathbb{C}}  \int_0^1 \int_0^1
    \overline{\mu(t)} \underbrace{\left( \sum_{i=1}^n q_i(t) \overline{q_i(s)} \right)}_{\coloneqq \, n \hat{C}(s,t)} \mu(s) \, dt ds
    \quad\text{s.t.}\,\, ||\mu|| = 1 \\
    =& \argmax_{\mu:[0,1]\rightarrow\mathbb{C}} \int_0^1
    \overline{\mu(t)} \int_0^1 \hat{C}(s,t) \mu(s) \, ds dt
    \quad\text{s.t.}\,\, ||\mu|| = 1
\end{align*}
with the solution given by the eigenfunction corresponding to the largest eigenvector of the complex empirical covariance function $\hat{C}(s,t) = n^{-1} \sum_{i=1}^n q_i(t) \overline{q_i(s})$.

\newpage
\subsection{The Full Procrustes Mean in a fixed basis}
To avoid having to sample the estimated covariance surface $\hat{C}(s,t)$ on a large grid when calculating its leading eigenfunction, it might be preferable to calculate this eigenfunction from the vector of basis coefficients directly.
After choosing a basis representation $b = (b_1, \dots, b_k)$ with $b_j : \mathbb{R} \rightarrow \mathbb{R}$ real-valued basis functions, we want to estimate complex coefficients $\theta_j \in \mathbb{C}$ so that the Full Procrustes mean of SRV curves is given by $\hat{\mu}(t) = \sum_{j=1}^k \hat{\theta}_j b_j(t) = b^T \hat{\theta}$:
\begin{align*}
    \hat{\mu} =& \argmax_{\theta : ||b^T\theta||=1} \sum_{i=1}^n \langle b^T\theta, q_i \rangle \langle q_i, b^T\theta \rangle \\
    =& \argmax_{\theta : ||b^T\theta||=1} \sum_{k,l} \sum_{i=1}^n \langle b_k \theta_k, q_i \rangle \langle q_i, b_l \theta_l \rangle \\
    =& \argmax_{\theta : ||b^T\theta||=1} \sum_{k,l} \bar{\theta}_k \theta_l \sum_{i=1}^n \langle b_k, q_i \rangle \langle q_i, b_l \rangle \\
    =& \argmax_{\theta : ||b^T\theta||=1} \theta^H S \theta \\
\end{align*}
where the matrix $S = \left\{ \sum_{i=1}^n \langle b_k, q_i \rangle \langle q_i, b_l \rangle \right\}_{k,l}$ has to be estimated from the observed SRV curves.
We can further simplify $S$ to
\begin{align*}
    S_{kl} =& \sum_{i=1}^n \int_0^1 \bar{b}_k(t) q_i(t) dt \int_0^1 \bar{q}_i(s) b_l(s) ds \\
    =& \int_0^1 \int_0^1 \bar{b}_k(t) \underbrace{\left( \sum_{i=1}^n q_i(t) \bar{q}_i(s) \right)}_{= n \, \hat{C}(s,t)} b_l(s) ds dt\\
    =& n \, \int_0^1 \int_0^1 \bar{b}_k(t) \hat{C}(s,t) b_l(s) ds dt\\
\end{align*}
with $\hat{C}(s,t) = \frac{1}{n} \sum_{i=1}^n q_i(s) \overline{q_i(t)}$ the sample analogue to the complex population covariance function $C(s,t) = \mathbb{E}[q(s)\overline{q(t)}]$.
We may estimate $C(s,t)$ via tensor product splines, so that $\hat{C}(s,t) = \sum_{k,l} \hat{\xi}_{kl} b_k(t) b_l(s)$, where $b_j(t)$, $j=1,\dots,k$ are the same real valued basis functions as used for the mean and $\hat{\xi}_{kl}$ are the estimated complex coefficients.
We can then further simplify $S_{kl}$
\begin{align*}
    S_{kl} =& n \, \int_0^1 \int_0^1 b_k(t) \left( \sum_{p,q} \hat{\xi}_{pq} b_q(t) b_p(s) \right) b_l(s) ds dt\\
    =& n \, \sum_{p,q} \hat{\xi}_{pq} \int_0^1 \int_0^1 b_k(t) b_q(t) b_p(s) b_l(s) ds dt\\
    =& n \, \sum_{p,q} \hat{\xi}_{pq} \langle b_k, b_q \rangle \langle b_p, b_l \rangle\\
    =& n \, \sum_{p,q} \hat{\xi}_{pq} g_{kq} g_{pl}
\end{align*}
where $g_{ij}$, $i,j = 1, \dots, k$ are the elements of the Gram matrix $G = bb^T$ with $G = \mathbb{I}_k$ in the special case of an orthogonal basis.
We can then write the write the matrix $S$ as a function of the estimated coefficient matrix $\hat{\Xi} = (\hat{\xi}_{ij})_{i,j = 1, \dots, k}$ :
\begin{align*}
    S =& n \, G \hat{\Xi} G
\end{align*}
The full Procrustes mean of SRV curves is then given by the solution to the optimization problem
\begin{align*}
    \hat{\mu} =& \argmax_{\theta} n \, \theta^H G \hat{\Xi} G \theta \quad \text{subj. to} \quad ||b^T \theta|| = 1 \\
    =& \argmax_{\theta : ||b^T\theta||=1} \, \theta^H G \hat{\Xi} G \theta \quad \text{subj. to} \quad \theta^H G \theta = 1
\end{align*}
One may solve this by using Lagrange optimization with the Langrangian
\begin{align*}
  \mathcal{L}(\theta,\lambda) =& \, \theta^H G \hat{\Xi} G \theta - \lambda ( \theta^H G \theta - 1)
\end{align*}


\newpage
\subsection{Estimation of the covariance surface $C(s,t)$}
Consider the following model for independent curves
\begin{equation}
    Y_i(t_{ij}) = \mu(t_{ij}, \mathbf{x}_i) + E_i(t_{ij}) + \epsilon(t_{ij}),
    \quad j = 1,\dots,D_i, \, i = 1,\dots,n,
\end{equation}
\textbf{[Fast symmetric additive cov smoothing, skew-symmetry, population vs. sample, etc.]}




\newpage
\section{Estimation Strategy}
\label{sec:algo}

\newpage
\section{Empirical Application}
\label{sec:app}

\newpage
\section{Outlook}
\label{sec:outl}

\newpage
\section{Summary}
\label{sec:sum}

\newpage
\nocite{*}
\printbibliography[heading=bibintoc] % Insert bilbiography and create toc entry


% --- Appendix

\newpage
\appendix
\addtocontents{toc}{\protect\setcounter{tocdepth}{1}}
\section{Derivations and Proofs}
\label{app:deriv}
\subsection{Derivation of the Full Procrustes Distance for Functional Data}
\label{app:deriv-full-proc-dist}
Consider two curves $\beta_1, \beta_2 : [0,1] \rightarrow \mathbb{C}$ with $\langle \beta_1, \mathbb{1} \rangle = \langle \beta_2, \mathbb{1} \rangle = 0$ where $\mathbb{1}$ is the constant function $\mathbb{1}(t) = 1$ for all $t \in [0,1]$.
Then $\beta_1$ and $\beta_2$ can be considered to be centered as
$$ \langle \beta_1, \mathbb{1} \rangle
= \int_0^1 \bar{\beta_1}(t)\mathbb{1}(t)dt\\
= \int_0^1 \bar{\beta_1}(t)dt
= \int_0^1 (y(t) + i x(t))dt
= \underbrace{\int_0^1 y(t)dt}_{\stackrel{!}{=}0} + i \underbrace{\int_0^1 x(t)dt}_{\stackrel{!}{=}0} = 0 $$

Then the full procrustes distance of $\beta_1, \beta_2$ is given by their minimum distance controlling for translation $\gamma \in \mathbb{C}$, and scaling and rotation $\omega = b e^{i\theta} \in \mathbb{C}$:
\begin{align*}
    d_F^2 =& \min_{\omega,\gamma \in \mathbb{C}} ||\beta_1 - \gamma \mathbb{1} - \omega \beta_2 ||^2 \\
    =& \min_{\omega,\gamma \in \mathbb{C}} \langle\beta_1 - \gamma \mathbb{1} - \omega \beta_2, \beta_1 - \gamma \mathbb{1} - \omega \beta_2 \rangle \\
    =& \min_{\omega,\gamma \in \mathbb{C}}
    \langle \beta_1 - \omega \beta_2, \beta_1 - \omega \beta_2 \rangle
    - \underbrace{\langle \beta_1, \gamma \mathbb{1} \rangle}_{=\,0}
    - \underbrace{\langle \gamma \mathbb{1}, \beta_1 \rangle}_{=\,0}
    + \underbrace{\langle \gamma \mathbb{1}, \omega \beta_2 \rangle}_{=\,0}
    + \underbrace{\langle \omega \beta_2, \gamma \mathbb{1} \rangle}_{=\,0}
    + \underbrace{\langle \gamma \mathbb{1}, \gamma \mathbb{1} \rangle}_{=||\gamma \mathbb{1}||^2} \\
    \stackrel{\gamma=0}{=}& \min_{\omega \in \mathbb{C}}\,
    \langle \beta_1, \beta_1 \rangle
    + \langle \omega \beta_2, \omega \beta_2 \rangle
    - \langle \beta_1, \omega \beta_2 \rangle
    - \langle \omega \beta_2, \beta_1 \rangle \\
    =& \min_{\omega \in \mathbb{C}}\,
    \langle \beta_1, \beta_1 \rangle
    + |\omega|^2 \langle \beta_2, \beta_2 \rangle
    - \omega \langle \beta_1, \beta_2 \rangle
    - \overline{\omega} \langle \beta_2, \beta_1 \rangle
\end{align*}
To find $\omega \in \mathbb{C}$ that minimizes $||\beta_1 - \omega \beta_2||^2$ we first consider the part of the problem dependent on $\theta$. We need to solve
$$ \min_{\omega \in \mathbb{C}}
- \omega \langle \beta_1, \beta_2 \rangle
- \overline{\omega} \langle \beta_2, \beta_1 \rangle  =
\max_{\omega \in \mathbb{C}}
\omega \langle \beta_1, \beta_2 \rangle
+ \overline{\omega} \langle \beta_2, \beta_1 \rangle $$
by using $\omega = b e^{i\theta}$ and $\langle \beta_1, \beta_2 \rangle = a e^{i\phi}$:
\begin{align*}
    \max_{\omega \in \mathbb{C}}
    \omega \langle \beta_1, \beta_2 \rangle
    + \overline{\omega} \langle \beta_2, \beta_1 \rangle
    =& \max_{b \in \mathbb{R}^+, \theta \in [0,2\pi]}
    b e^{i\theta} a e^{i\phi} + b e^{-i\theta} a e^{-i\phi} \\
    =& \max_{b \in \mathbb{R}^+, \theta \in [0,2\pi]}
    b e^{i\theta} a e^{i\phi} + b e^{-i\theta} a e^{-i\phi} \\
    =& \max_{b \in \mathbb{R}^+, \theta \in [0,2\pi]}
    2 b a \cos\left(\theta +\phi\right)\\
    \stackrel{\theta = -\phi}{=}& \max_{b \in \mathbb{R}^+}
    2 b a
\end{align*}
and using $\theta = -\phi$ the original mimization problem therefore simplifies to
\begin{align*}
    d_F^2 =&
    \min_{b \in \mathbb{R}^+}\,
    \langle \beta_1, \beta_1 \rangle
    + b^2 \langle \beta_2, \beta_2 \rangle
    - 2 b a \\
    \frac{\partial d_F^2}{\partial b} =& \,
    2 b \langle \beta_2, \beta_2 \rangle - 2a \stackrel{!}{=} 0 \\
    \Rightarrow \quad & b = \frac{a}{\langle \beta_2, \beta_2 \rangle}
\end{align*}
And for the \textit{full Procrustes distance} it follows that
$$ d_F^2
= \langle \beta_1, \beta_1 \rangle - \frac{a^2}{\langle \beta_2, \beta_2 \rangle}
= \langle \beta_1, \beta_1 \rangle - \frac{ \langle \beta_1, \beta_2 \rangle \langle \beta_2, \beta_1 \rangle}{\langle \beta_2, \beta_2 \rangle}$$
As this expression is not symmetric in $\beta_1$ and $\beta_2$ we can take the curves to be of unit length with $\tilde{\beta}_j = \frac{\beta_j}{||\beta_j||}$, $j=1,2$ with $||\beta_j|| = \sqrt{\langle \beta_j, \beta_j \rangle}$, so that $\langle \tilde\beta_1, \tilde\beta_1 \rangle = \langle \tilde\beta_2, \tilde\beta_2 \rangle = 1$ and obtain a suitable measure of distance:
$$ d_F = \sqrt{1 - \langle \tilde\beta_1, \tilde\beta_2 \rangle
\langle \tilde\beta_2, \tilde\beta_1 \rangle}
= \sqrt{1 - \frac{ \langle \beta_1, \beta_2 \rangle \langle \beta_2, \beta_1 \rangle}{\langle \beta_1, \beta_1 \rangle \langle \beta_2, \beta_2 \rangle}}$$




\newpage
\section{Supplementary Materials}
\label{app:deriv}
\subsection{Implementation Notes}

\subsection{Replication Guide}


\newpage
\section{---Discarded---}
\section*{Math-Basics Recap}
\subsection*{Scalar Products}
$V$ n-dimensional vector space with basis $B = (b_1, \dots, b_n)$, then any scalar product $\langle\cdot,\cdot\rangle$ on $V$ can be expressed using a $(n \times n)$ matrix $G$, the Gram matrix of the scalar product. Its entries are the scalar products of the basis vectors:
$$ G = (g_{ij})_{i,j=1,\dots,n} \quad \text{with} 
  \quad g_{ij} = \langle b_i, b_j \rangle \quad \text{for}
  \quad i,j = 1,\dots, n $$
When vectors $x,y \in V$ are expressed with respect to the basis $B$ as
$$ x = \sum_{i=1}^n x_i b_i \quad \text{and} \quad y = \sum_{i=1}^n y_i b_i $$
the scalar product can be expressed using the Gram matrix, and in the complex case it holds that
$$ \langle x, y \rangle = \sum^n_{i,j=1} \bar{x}_i y_j \langle b_i, b_j \rangle 
  =\sum^n_{i,j=1} \bar{x}_i g_{ij} y_j = x^\dagger G y$$
when $x_i,y_i \in \mathbb{C}$ for $i=1,\dots,n$ with $x^\dagger$ indicating the conjugate transpose of $x = (x_1, \dots, x_n)^T$. If $B$ is an \textit{orthonormal} basis, that is if $\langle b_i, b_j \rangle = \delta_{ij}$, it further holds that $\langle x,y \rangle = x^\dagger y$ as $G = \mathbb{1}_{n \times n}$.


\subsection*{Functional Scalar Products}
This concept can be generalized for vectors in function spaces. Define the scalar product of two functions $f(t), g(t)$ as:
$$ \langle f, g \rangle = \int_a^b \bar{f}(t) w(t) g(t) dt $$
with weighting function $w(t)$ and $[a,b]$ depending on the function space. The scalar product has the following properties:
\begin{enumerate}
    \item $\langle f, g + h \rangle = \langle f,g \rangle + \langle f,h \rangle$
    \item $\langle f, g \rangle = \overline{\langle g, f \rangle}$
    \item $\langle f, cg \rangle = c \langle f,g \rangle$ or, using (2),
        $\langle cf,g \rangle = \bar{c} \langle f,g \rangle$ for $c \in \mathbb{C}$
\end{enumerate}
If we have a functional basis $\{\phi_1, \dots , \phi_n\}$ (and possibly $n \to \infty$) of our function space we can also write the function $f$ as an expansion
$$ f = \sum_{i=1}^n a_i \phi_i \quad \text{so that} \quad
  f(t) = \sum_{i=1}^n a_i \phi_i (t)$$
Additionally, if we have a \textit{orthogonal} basis, so that $\langle \phi_i, \phi_j \rangle = 0$ for $i \neq j$, we can take the scalar product with $\phi_k$ from the left
$$ \langle \phi_k, f \rangle = \sum_{i=1}^n a_i \langle \phi_k, \phi_i \rangle =
  a_k \langle \phi_k, \phi_k \rangle $$
which yields the coefficients $a_k$: 
$$ a_k = \frac{\langle \phi_k, f \rangle}{\langle \phi_k, \phi_k \rangle}$$

For an \textit{orthonormal} basis it holds that $\langle \phi_i, \phi_j \rangle = \delta_{ij}$. Suppose that two functions $f,g$ are expanded in the same orthonormal basis:
$$ f = \sum_{i=1}^n a_i \phi_i \quad \text{and} \quad 
  g = \sum_{i=1}^n b_i \phi_i $$
We can then write the scalar product as:
$$ \langle f,g \rangle = 
  \langle \sum_{i=1}^n a_i \phi_i, \sum_{i=1}^n b_i \phi_i \rangle = 
  \sum_{i=1}^n \sum_{j=1}^n \hat{a}_i b_j \langle \phi_i, \phi_j \rangle =
  \sum_{i=1}^n \bar{a}_i b_i = a^\dagger b$$
for coefficient vectors $a, b \in \mathbb{C}^n$. This means that the functional scalar product reduces to a complex dot product. Additionally it holds that for the norm $||\cdot||$ of a function $f$:
$$ ||f|| = \langle f,f \rangle^{\frac{1}{2}} = 
  \sqrt{a^\dagger a} = \sqrt{\sum_{i=1}^n |a_i|^2}$$

\section*{FDA-Basics Recap}
As discussed in the last section we can express a function $f$ in its \textit{basis function expansion} using a set of basis functions $\phi_k$ with $k=1,\dots,K$ and a set of coefficients $c_1,\dots,c_K$ (both possibly $\mathbb{C}$ valued e.g.\ in the case of $2D$-curves)
$$ f = \sum_{k=1}^K c_k \phi_k = \bm{c'}\bm{\phi} $$
where in the matrix notation $\bm{c}$ and $\bm{\phi}$ are the vectors containing the coefficients and basis functions.

When considering a sample of $N$ functions $f_i$ we can write this in matrix notation as 
$$ \bm{f} = \bm{C}\bm{\phi} $$
where $C$ is a $(N \times K)$ matrix of coefficients and $\bm{f}$ is a vector containing the $N$ functions.

\subsection*{Smoothing by Regression}
When working with functional data we can usually never observe a function $f$ directly and instead only observe discrete points $(x_i, t_i)$ along the curve, with $f(t_i) = x_i$.
As we don't know the exact functional form of $f$, calculating the scalar products $\langle \phi_k, f \rangle$ and therefore calculating the coefficients $c_k$ of a given basis representation is not possible.

However, we can estimate the basis coefficients using e.g.\ regression analysis an approach motivated by the error model
$$ f(t_i) = \bm{c'}\bm{\phi(t_i)} + \epsilon_i $$
If we observe our function $n$ times at $t_1,\dots,t_n$, we can estimate the coefficients from a least squares problem, where we try to minimize the deviation of the basis expansion from the observed values.
Using matrix notation let the vector $\bm{f}$ contains the observed values $f(t_i)$, $i=1,\dots,n$ and $(n \times k)$ matrix $\bm{\Phi}$ contains the basis function values $\phi_k(t_i)$.
Then we have
$$ \bm{f} = \bm{\Phi}\bm{c} + \bm{\epsilon} $$
with the estimate for the coefficient vector $\bm{c}$ given by
$$ \hat{\bm{c}} = \left( \bm{\Phi'} \bm{\Phi}\right)^{-1} \bm{\Phi'} \bm{f}. $$
Spline curves fit in this way are often called \textit{regression splines}.


\subsection*{Common Basis Representations}
\paragraph{Piecewise Polynomials (Splines)}
Splines are defined by their range of validity, the knots, and the order.
Their are constructed by dividing the area of observation into subintervals with boundaries at points called \textit{breaks}.
Over any subinterval the spline function is a polynomial of fixed degree or order.
The term \textit{degree} refers to the highest power in the polynomial while its \textit{order} is one higher than its degree.
E.g.\ a line has degree one but order two because it also has a constant term. 
\textbf{[\dots]}


\paragraph{Polygonal Basis}
\textbf{[\dots]}


\subsection*{Bivariate Functional Data}
The analogue of covariance matrices in MVA are covariance surfaces $\sigma(s,t)$ whose values specify the covariance between values $f(s)$ and $f(t)$ over a population of curves.
We can write these bivariate functions in a \textit{bivariate basis expansion} $$ r(s,t) = \sum_{k=1}^K \sum_{l=1}^K b_{k,l} \phi_k(s) \psi_l(t) 
  = \bm{\phi}(s)' \bm{B} \bm{\psi}(t) $$
with a $K \times K$ coefficient matrix $B$ and two sets of basis functions $\phi_k$ and $\psi_l$ using \textit{Tensor Product Splines}
$$ B_{k,l}(s,t) = \phi_k(s) \psi_l(t).$$





\end{document}

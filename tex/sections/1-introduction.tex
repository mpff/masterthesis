\subsection{Scalar Products}
$V$ n-dimensional vector space with basis $B = (b_1, \dots, b_n)$, then any scalar product $\langle\cdot,\cdot\rangle$ on $V$ can be expressed using a $(n \times n)$ matrix $G$, the Gram matrix of the scalar product. Its entries are the scalar products of the basis vectors:
$$ G = (g_{ij})_{i,j=1,\dots,n} \quad \text{with} 
  \quad g_{ij} = \langle b_i, b_j \rangle \quad \text{for}
  \quad i,j = 1,\dots, n $$
When vectors $x,y \in V$ are expressed with respect to the basis $B$ as
$$ x = \sum_{i=1}^n x_i b_i \quad \text{and} \quad y = \sum_{i=1}^n y_i b_i $$
the scalar product can be expressed using the Gram matrix, and in the complex case it holds that
$$ \langle x, y \rangle = \sum^n_{i,j=1} \bar{x}_i y_j \langle b_i, b_j \rangle 
  =\sum^n_{i,j=1} \bar{x}_i g_{ij} y_j = x^\dagger G y$$
when $x_i,y_i \in \mathbb{C}$ for $i=1,\dots,n$ with $x^\dagger$ indicating the conjugate transpose of $x = (x_1, \dots, x_n)^T$. If $B$ is an \textit{orthonormal} basis, that is if $\langle b_i, b_j \rangle = \delta_{ij}$, it further holds that $\langle x,y \rangle = x^\dagger y$ as $G = \mathbb{1}_{n \times n}$.


\subsection{Functional Scalar Products}
This concept can be generalized for vectors in function spaces. Define the scalar product of two functions $f(t), g(t)$ as:
$$ \langle f, g \rangle = \int_a^b \bar{f}(t) w(t) g(t) dt $$
with weighting function $w(t)$ and $[a,b]$ depending on the function space. The scalar product has the following properties:
\begin{enumerate}
    \item $\langle f, g + h \rangle = \langle f,g \rangle + \langle f,h \rangle$
    \item $\langle f, g \rangle = \overline{\langle g, f \rangle}$
    \item $\langle f, cg \rangle = c \langle f,g \rangle$ or, using (2),
        $\langle cf,g \rangle = \bar{c} \langle f,g \rangle$ for $c \in \mathbb{C}$
\end{enumerate}
If we have a functional basis $\{\phi_1, \dots , \phi_n\}$ (and possibly $n \to \infty$) of our function space we can also write the function $f$ as an expansion
$$ f = \sum_{i=1}^n a_i \phi_i \quad \text{so that} \quad
  f(t) = \sum_{i=1}^n a_i \phi_i (t)$$
Additionally, if we have a \textit{orthongonal} basis, so that $\langle \phi_i, \phi_j \rangle = 0$ for $i \neq j$, we can take the scalar product with $\phi_k$ from the left
$$ \langle \phi_k, f \rangle = \sum_{i=1}^n a_i \langle \phi_k, \phi_i \rangle =
  a_k \langle \phi_k, \phi_k \rangle $$
which yields the coefficients $a_k$: 
$$ a_k = \frac{\langle \phi_k, f \rangle}{\langle \phi_k, \phi_k \rangle}$$

For an \textit{orthonormal} basis it holds that $\langle \phi_i, \phi_j \rangle = \delta_{ij}$. Suppose that two functions $f,g$ are expanded in the same orthonormal basis:
$$ f = \sum_{i=1}^n a_i \phi_i \quad \text{and} \quad 
  g = \sum_{i=1}^n b_i \phi_i $$
We can then write the scalar product as:
$$ \langle f,g \rangle = 
  \langle \sum_{i=1}^n a_i \phi_i, \sum_{i=1}^n b_i \phi_i \rangle = 
  \sum_{i=1}^n \sum_{j=1}^n \hat{a}_i b_j \langle \phi_i, \phi_j \rangle =
  \sum_{i=1}^n \bar{a}_i b_i = a^\dagger b$$
for coefficient vectors $a, b \in \mathbb{C}^n$. This means that the functional scalar product reduces to a complex dot product. Additionally it holds that for the norm $||\cdot||$ of a function $f$:
$$ ||f|| = \langle f,f \rangle^{\frac{1}{2}} = 
  \sqrt{a^\dagger a} = \sqrt{\sum_{i=1}^n |a_i|^2}$$

\section{FDA-Basics Recap}
As discussed in the last section we can express a function $f$ in its \textit{basis function expansion} using a set of basis functions $\phi_k$ with $k=1,\dots,K$ and a set of coefficients $c_1,\dots,c_K$ (both possibly $\mathbb{C}$ valued e.g.\ in the case of $2D$-curves)
$$ f = \sum_{k=1}^K c_k \phi_k = \bm{c'}\bm{\phi} $$
where in the matrix notation $\bm{c}$ and $\bm{\phi}$ are the vectors containing the coefficients and basis functions.

When considering a sample of $N$ functions $f_i$ we can write this in matrix notation as 
$$ \bm{f} = \bm{C}\bm{\phi} $$
where $C$ is a $(N \times K)$ matrix of coefficients and $\bm{f}$ is a vector containing the $N$ functions.

\subsection{Smoothing by Regression}
When working with functional data we can usually never observe a function $f$ directly and instead only observe discrete points $(x_i, t_i)$ along the curve, with $f(t_i) = x_i$.
As we don't know the exact functional form of $f$, calculating the scalar products $\langle \phi_k, f \rangle$ and therefore calculating the coefficients $c_k$ of a given basis representation is not possible.

However, we can estimate the basis coefficients using e.g.\ regression analysis an approach motivated by the error model
$$ f(t_i) = \bm{c'}\bm{\phi(t_i)} + \epsilon_i $$
If we observe our function $n$ times at $t_1,\dots,t_n$, we can estimate the coefficients from a least squares problem, where we try to minimize the deviation of the basis expansion from the observed values.
Using matrix notation let the vector $\bm{f}$ contains the observed values $f(t_i)$, $i=1,\dots,n$ and $(n \times k)$ matrix $\bm{\Phi}$ contains the basis function values $\phi_k(t_i)$.
Then we have
$$ \bm{f} = \bm{\Phi}\bm{c} + \bm{\epsilon} $$
with the estimate for the coefficient vector $\bm{c}$ given by
$$ \hat{\bm{c}} = \left( \bm{\Phi'} \bm{\Phi}\right)^{-1} \bm{\Phi'} \bm{f}. $$
Spline curves fit in this way are often called \textit{regression splines}.


\subsection{Common Basis Representations}
\paragraph{Piecewise Polynomials (Splines)}
Splines are defined by their range of validity, the knots, and the order.
Their are constructed by dividing the area of observation into subintervals with boundaries at points called \textit{breaks}.
Over any subinterval the spline function is a polynomial of fixed degree or order.
The term \textit{degree} refers to the highest power in the polynomial while its \textit{order} is one higher than its degree.
E.g.\ a line has degree one but order two because it also has a constant term. 
\textbf{[\dots]}


\paragraph{Polygonal Basis}
\textbf{[\dots]}


\subsection{Bivariate Functional Data}
The analogue of covariance matrices in MVA are covariance surfaces $\sigma(s,t)$ whose values specify the covariance between values $f(s)$ and $f(t)$ over a population of curves.
We can write these bivariate functions in a \textit{bivariate basis expansion} $$ r(s,t) = \sum_{k=1}^K \sum_{l=1}^K b_{k,l} \phi_k(s) \psi_l(t) 
  = \bm{\phi}(s)' \bm{B} \bm{\psi}(t) $$
with a $K \times K$ coefficient matrix $B$ and two sets of basis functions $\phi_k$ and $\psi_l$ using \textit{Tensor Product Splines}
$$ B_{k,l}(s,t) = \phi_k(s) \psi_l(t).$$



\newpage
\section{The Full Procrustes Mean for Planar Curves}
Let $\beta$ be a continuous planar curve. It can be represented in a parameterized form in $\mathbb{R}^2$ as
$$ \beta : [0,1] \rightarrow \mathbb{R}^2,\quad \beta(t) = ( x(t), y(t)) \,, $$
where $x, y$ are scalar-valued \textit{coordinate functions} of $\beta$ and $t$ is the parameter.
We can similarly represent it using the complex numbers $\mathbb{C}$ as
$$ \beta : [0,1] \rightarrow \mathbb{C},\quad \beta(t) = x(t) + iy(t) \,, $$
with the benefit that complex notation is often much simpler to use in the 2D case.

For a set of planar curves $\beta_1,\dots,\beta_n : [0,1] \rightarrow \mathbb{C}$ the \textit{full Procrustes mean} $\hat{\mu}$ is defined as the curve minimizing the sum of squared \textit{full Procrustes distances} from each $y_i$ to an unknown unit size mean configuration $\mu$, that is
$$ \hat{\mu} = \argmin_{\mu:[0,1]\rightarrow\mathbb{C}} \sum_{i=1}^n d^2_F(\mu,¸\beta_i) 
  \quad\text{s.t.}\,\, ||\mu||^2 = 1\,.$$

\subsection{The Full Procrustes Distance}
Consider two curves $\beta_1, \beta_2 : [0,1] \rightarrow \mathbb{C}$ with $\langle \beta_1, \mathbb{1} \rangle = \langle \beta_2, \mathbb{1} \rangle = 0$ where $\mathbb{1}$ is the constant function $\mathbb{1}(t) = 1$ for all $t \in [0,1]$.
Then $\beta_1$ and $\beta_2$ can be considered to be centered as
$$ \langle \beta_1, \mathbb{1} \rangle 
  = \int_0^1 \bar{\beta_1}(t)\mathbb{1}(t)dt\\
  = \int_0^1 \bar{\beta_1}(t)dt 
  = \int_0^1 (y(t) + i x(t))dt
  = \underbrace{\int_0^1 y(t)dt}_{\stackrel{!}{=}0} + i \underbrace{\int_0^1 x(t)dt}_{\stackrel{!}{=}0} = 0 $$

Then the full procrustes distance of $\beta_1, \beta_2$ is given by their minimum distance controlling for translation $\gamma \in \mathbb{C}$, and scaling and rotation $\omega = b e^{i\theta} \in \mathbb{C}$:
\begin{align*}
d_F^2 =& \min_{\omega,\gamma \in \mathbb{C}} ||\beta_1 - \gamma \mathbb{1} - \omega \beta_2 ||^2 \\
  =& \min_{\omega,\gamma \in \mathbb{C}} \langle\beta_1 - \gamma \mathbb{1} - \omega \beta_2, \beta_1 - \gamma \mathbb{1} - \omega \beta_2 \rangle \\
  =& \min_{\omega,\gamma \in \mathbb{C}}
    \langle \beta_1 - \omega \beta_2, \beta_1 - \omega \beta_2 \rangle
    - \underbrace{\langle \beta_1, \gamma \mathbb{1} \rangle}_{=\,0}
    - \underbrace{\langle \gamma \mathbb{1}, \beta_1 \rangle}_{=\,0}
    + \underbrace{\langle \gamma \mathbb{1}, \omega \beta_2 \rangle}_{=\,0}
    + \underbrace{\langle \omega \beta_2, \gamma \mathbb{1} \rangle}_{=\,0}
    + \underbrace{\langle \gamma \mathbb{1}, \gamma \mathbb{1} \rangle}_{=||\gamma \mathbb{1}||^2} \\
  \stackrel{\gamma=0}{=}& \min_{\omega \in \mathbb{C}}\,
    \langle \beta_1, \beta_1 \rangle 
    + \langle \omega \beta_2, \omega \beta_2 \rangle
    - \langle \beta_1, \omega \beta_2 \rangle
    - \langle \omega \beta_2, \beta_1 \rangle \\
  =& \min_{\omega \in \mathbb{C}}\,
    \langle \beta_1, \beta_1 \rangle 
    + |\omega|^2 \langle \beta_2, \beta_2 \rangle
    - \omega \langle \beta_1, \beta_2 \rangle
    - \overline{\omega} \langle \beta_2, \beta_1 \rangle
\end{align*}
To find $\omega \in \mathbb{C}$ that minimizes $||\beta_1 - \omega \beta_2||^2$ we first consider the part of the problem dependent on $\theta$. We need to solve
$$ \min_{\omega \in \mathbb{C}} 
  - \omega \langle \beta_1, \beta_2 \rangle
  - \overline{\omega} \langle \beta_2, \beta_1 \rangle  = 
\max_{\omega \in \mathbb{C}}
  \omega \langle \beta_1, \beta_2 \rangle
  + \overline{\omega} \langle \beta_2, \beta_1 \rangle $$
by using $\omega = b e^{i\theta}$ and $\langle \beta_1, \beta_2 \rangle = a e^{i\phi}$:
\begin{align*}
\max_{\omega \in \mathbb{C}}
  \omega \langle \beta_1, \beta_2 \rangle
  + \overline{\omega} \langle \beta_2, \beta_1 \rangle
=& \max_{b \in \mathbb{R}^+, \theta \in [0,2\pi]}
  b e^{i\theta} a e^{i\phi} + b e^{-i\theta} a e^{-i\phi} \\
=& \max_{b \in \mathbb{R}^+, \theta \in [0,2\pi]}
  b e^{i\theta} a e^{i\phi} + b e^{-i\theta} a e^{-i\phi} \\
=& \max_{b \in \mathbb{R}^+, \theta \in [0,2\pi]}
 2 b a \cos\left(\theta +\phi\right)\\
\stackrel{\theta = -\phi}{=}& \max_{b \in \mathbb{R}^+}
 2 b a 
\end{align*}
and using $\theta = -\phi$ the original mimization problem therefore simplifies to
\begin{align*}
d_F^2 =& 
\min_{b \in \mathbb{R}^+}\,
  \langle \beta_1, \beta_1 \rangle 
  + b^2 \langle \beta_2, \beta_2 \rangle
  - 2 b a \\
\frac{\partial d_F^2}{\partial b} =& \,
  2 b \langle \beta_2, \beta_2 \rangle - 2a \stackrel{!}{=} 0 \\
\Rightarrow \quad & b = \frac{a}{\langle \beta_2, \beta_2 \rangle}
\end{align*}
And for the \textit{full Procrustes distance} it follows that
$$ d_F^2 
= \langle \beta_1, \beta_1 \rangle - \frac{a^2}{\langle \beta_2, \beta_2 \rangle}
= \langle \beta_1, \beta_1 \rangle - \frac{ \langle \beta_1, \beta_2 \rangle \langle \beta_2, \beta_1 \rangle}{\langle \beta_2, \beta_2 \rangle}$$
As this expression is not symmetric in $\beta_1$ and $\beta_2$ we can take the curves to be of unit length with $\tilde{\beta}_j = \frac{\beta_j}{||\beta_j||}$, $j=1,2$ with $||\beta_j|| = \sqrt{\langle \beta_j, \beta_j \rangle}$, so that $\langle \tilde\beta_1, \tilde\beta_1 \rangle = \langle \tilde\beta_2, \tilde\beta_2 \rangle = 1$ and obtain a suitable measure of distance:
$$ d_F = \sqrt{1 - \langle \tilde\beta_1, \tilde\beta_2 \rangle
                   \langle \tilde\beta_2, \tilde\beta_1 \rangle}
       = \sqrt{1 - \frac{ \langle \beta_1, \beta_2 \rangle \langle \beta_2, \beta_1 \rangle}{\langle \beta_1, \beta_1 \rangle \langle \beta_2, \beta_2 \rangle}}$$




\subsection{The Full Procrustes mean}
Again, for a set of planar curves $\beta_1,\dots,\beta_n : [0,1] \rightarrow \mathbb{C}$ the \textit{full Procrustes mean} $\hat{\mu}$ is defined as the curve minimizing the sum of squared \textit{full Procrustes distances} from each $\beta_i$ to an unknown unit size mean configuration $\mu$, that is
\begin{align*}
  \hat{\mu} =& \argmin_{\mu:[0,1]\rightarrow\mathbb{C}} \sum_{i=1}^n d^2_F(\mu,\beta_i) 
    \quad\text{s.t.}\,\, ||\mu|| = 1 \\
  =& \argmin_{\mu:[0,1]\rightarrow\mathbb{C}} \sum_{i=1}^n 1 - \frac{\langle \mu, \beta_i \rangle \langle \beta_i, \mu \rangle}{\langle \mu, \mu \rangle \langle \beta_i, \beta_i \rangle}
    \quad\text{s.t.}\,\, ||\mu|| = 1 \\
  =& \argmax_{\mu:[0,1]\rightarrow\mathbb{C}} \sum_{i=1}^n \langle \mu, \beta_i \rangle \langle \beta_i, \mu \rangle
    \quad\text{s.t.}\,\, ||\mu|| = 1 \\
  =& \argmax_{\mu:[0,1]\rightarrow\mathbb{C}} \sum_{i=1}^n 
    \int_0^1 \bar\mu(t) \beta_i(t) dt \int_0^1 \bar\beta_i(s) \mu(s) ds
    \quad\text{s.t.}\,\, ||\mu|| = 1 \\
  =& \argmax_{\mu:[0,1]\rightarrow\mathbb{C}}  \int_0^1 \int_0^1 
    \bar\mu(t) \underbrace{\left( \sum_{i=1}^n \beta_i(t) \bar\beta_i(s) \right)}_{\coloneqq \hat{C}(s,t)} \mu(s) dt ds
    \quad\text{s.t.}\,\, ||\mu|| = 1 \\
  =& \argmax_{\mu:[0,1]\rightarrow\mathbb{C}} \int_0^1 \bar\mu(t)
  \underbrace{\int_0^1 \hat{C}(s,t) \mu(s) ds}_{\coloneqq \hat{A\mu}(t)} dt
    \quad\text{s.t.}\,\, ||\mu|| = 1 \\
  =& \argmax_{\mu:[0,1]\rightarrow\mathbb{C}}
    \langle \mu, \hat{A\mu} \rangle
    \quad\text{s.t.}\,\, ||\mu|| = 1 \\
\end{align*}
and using an orthonormal basis expansion $\hat{C}(s,t) = \sum_{k=1}^\infty \hat{c}_k \hat{\psi}_k(s) \hat{\psi}_k(t)$ \textbf{[Note: not sure about this...]}
\begin{align*}
  \hat{\mu}=& \argmax_{\mu:[0,1]\rightarrow\mathbb{C}}  \sum_{k=1}^\infty \int_0^1 \int_0^1 
    \hat{c}_k \bar\mu(t) \hat\psi_k(t) \mu(s) \hat\psi_k(s) dt ds
    \quad\text{s.t.}\,\, ||\mu|| = 1 \\
  =& \argmax_{\mu:[0,1]\rightarrow\mathbb{C}}  \sum_{k=1}^\infty  
    \hat{c}_k \langle\mu, \hat\psi_k \rangle \langle \bar\mu, \hat\psi_k \rangle
    \quad\text{s.t.}\,\, ||\mu|| = 1 
\end{align*}


\textbf{[\dots]}

When dealing with sparsely and irregularly sampled curves the calculation of the scalar products above becomes challenging.
We can instead treat the empirical estimator above as an estimator for the 'population level' full Procrustes mean
\begin{align*}
\hat{\mu} =& \argmax_{\mu:||\mu||=1} \mathbb{E}[\langle \mu, B \rangle \langle B,\mu \rangle]\\
  =& \argmax_{\mu:||\mu||=1}
\end{align*}

\subsection{Estimation of the covariance surface $C(s,t)$}
Consider the following model for independent curves
\begin{equation}
  Y_i(t_{ij}) = \mu(t_{ij}, \mathbf{x}_i) + E_i(t_{ij}) + \epsilon(t_{ij}),
    \quad j = 1,\dots,D_i, \, i = 1,\dots,n,
\end{equation}

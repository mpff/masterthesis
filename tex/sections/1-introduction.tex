\subsection{Scalar Products}
$V$ n-dimensional vector space with basis $B = (b_1, \dots, b_n)$, then any scalar product $\langle\cdot,\cdot\rangle$ on $V$ can be expressed using a $(n \times n)$ matrix $G$, the Gram matrix of the scalar product. Its entries are the scalar products of the basis vectors:
$$ G = (g_{ij})_{i,j=1,\dots,n} \quad \text{with} 
  \quad g_{ij} = \langle b_i, b_j \rangle \quad \text{for}
  \quad i,j = 1,\dots, n $$
When vectors $x,y \in V$ are expressed with respect to the basis $B$ as
$$ x = \sum_{i=1}^n x_i b_i \quad \text{and} \quad y = \sum_{i=1}^n y_i b_i $$
the scalar product can be expressed using the Gram matrix, and in the complex case it holds that
$$ \langle x, y \rangle = \sum^n_{i,j=1} \bar{x}_i y_j \langle b_i, b_j \rangle 
  =\sum^n_{i,j=1} \bar{x}_i g_{ij} y_j = x^\dagger G y$$
when $x_i,y_i \in \mathbb{C}$ for $i=1,\dots,n$ with $x^\dagger$ indicating the conjugate transpose of $x = (x_1, \dots, x_n)^T$. If $B$ is an \textit{orthonormal} basis, that is if $\langle b_i, b_j \rangle = \delta_{ij}$, it further holds that $\langle x,y \rangle = x^\dagger y$ as $G = \mathbb{1}_{n \times n}$.


\subsection{Functional Scalar Products}
This concept can be generalized for vectors in function spaces. Define the scalar product of two functions $f(t), g(t)$ as:
$$ \langle f, g \rangle = \int_a^b \bar{f}(t) w(t) g(t) dt $$
with weighting function $w(t)$ and $[a,b]$ depending on the function space. The scalar product has the following properties:
\begin{enumerate}
    \item $\langle f, g + h \rangle = \langle f,g \rangle + \langle f,h \rangle$
    \item $\langle f, g \rangle = \overline{\langle g, f \rangle}$
    \item $\langle f, cg \rangle = c \langle f,g \rangle$ or, using (2),
        $\langle cf,g \rangle = \bar{c} \langle f,g \rangle$ for $c \in \mathbb{C}$
\end{enumerate}
If we have a functional basis $\{\phi_1, \dots , \phi_n\}$ (and possibly $n \to \infty$) of our function space we can also write the function $f$ as an expansion
$$ f = \sum_{i=1}^n a_i \phi_i \quad \text{so that} \quad
  f(t) = \sum_{i=1}^n a_i \phi_i (t)$$
Additionally, if we have a \textit{orthogonal} basis, so that $\langle \phi_i, \phi_j \rangle = 0$ for $i \neq j$, we can take the scalar product with $\phi_k$ from the left
$$ \langle \phi_k, f \rangle = \sum_{i=1}^n a_i \langle \phi_k, \phi_i \rangle =
  a_k \langle \phi_k, \phi_k \rangle $$
which yields the coefficients $a_k$: 
$$ a_k = \frac{\langle \phi_k, f \rangle}{\langle \phi_k, \phi_k \rangle}$$

For an \textit{orthonormal} basis it holds that $\langle \phi_i, \phi_j \rangle = \delta_{ij}$. Suppose that two functions $f,g$ are expanded in the same orthonormal basis:
$$ f = \sum_{i=1}^n a_i \phi_i \quad \text{and} \quad 
  g = \sum_{i=1}^n b_i \phi_i $$
We can then write the scalar product as:
$$ \langle f,g \rangle = 
  \langle \sum_{i=1}^n a_i \phi_i, \sum_{i=1}^n b_i \phi_i \rangle = 
  \sum_{i=1}^n \sum_{j=1}^n \hat{a}_i b_j \langle \phi_i, \phi_j \rangle =
  \sum_{i=1}^n \bar{a}_i b_i = a^\dagger b$$
for coefficient vectors $a, b \in \mathbb{C}^n$. This means that the functional scalar product reduces to a complex dot product. Additionally it holds that for the norm $||\cdot||$ of a function $f$:
$$ ||f|| = \langle f,f \rangle^{\frac{1}{2}} = 
  \sqrt{a^\dagger a} = \sqrt{\sum_{i=1}^n |a_i|^2}$$

\section{FDA-Basics Recap}
As discussed in the last section we can express a function $f$ in its \textit{basis function expansion} using a set of basis functions $\phi_k$ with $k=1,\dots,K$ and a set of coefficients $c_1,\dots,c_K$ (both possibly $\mathbb{C}$ valued e.g.\ in the case of $2D$-curves)
$$ f = \sum_{k=1}^K c_k \phi_k = \bm{c'}\bm{\phi} $$
where in the matrix notation $\bm{c}$ and $\bm{\phi}$ are the vectors containing the coefficients and basis functions.

When considering a sample of $N$ functions $f_i$ we can write this in matrix notation as 
$$ \bm{f} = \bm{C}\bm{\phi} $$
where $C$ is a $(N \times K)$ matrix of coefficients and $\bm{f}$ is a vector containing the $N$ functions.

\subsection{Smoothing by Regression}
When working with functional data we can usually never observe a function $f$ directly and instead only observe discrete points $(x_i, t_i)$ along the curve, with $f(t_i) = x_i$.
As we don't know the exact functional form of $f$, calculating the scalar products $\langle \phi_k, f \rangle$ and therefore calculating the coefficients $c_k$ of a given basis representation is not possible.

However, we can estimate the basis coefficients using e.g.\ regression analysis an approach motivated by the error model
$$ f(t_i) = \bm{c'}\bm{\phi(t_i)} + \epsilon_i $$
If we observe our function $n$ times at $t_1,\dots,t_n$, we can estimate the coefficients from a least squares problem, where we try to minimize the deviation of the basis expansion from the observed values.
Using matrix notation let the vector $\bm{f}$ contains the observed values $f(t_i)$, $i=1,\dots,n$ and $(n \times k)$ matrix $\bm{\Phi}$ contains the basis function values $\phi_k(t_i)$.
Then we have
$$ \bm{f} = \bm{\Phi}\bm{c} + \bm{\epsilon} $$
with the estimate for the coefficient vector $\bm{c}$ given by
$$ \hat{\bm{c}} = \left( \bm{\Phi'} \bm{\Phi}\right)^{-1} \bm{\Phi'} \bm{f}. $$
Spline curves fit in this way are often called \textit{regression splines}.


\subsection{Common Basis Representations}
\paragraph{Piecewise Polynomials (Splines)}
Splines are defined by their range of validity, the knots, and the order.
Their are constructed by dividing the area of observation into subintervals with boundaries at points called \textit{breaks}.
Over any subinterval the spline function is a polynomial of fixed degree or order.
The term \textit{degree} refers to the highest power in the polynomial while its \textit{order} is one higher than its degree.
E.g.\ a line has degree one but order two because it also has a constant term. 
\textbf{[\dots]}


\paragraph{Polygonal Basis}
\textbf{[\dots]}


\subsection{Bivariate Functional Data}
The analogue of covariance matrices in MVA are covariance surfaces $\sigma(s,t)$ whose values specify the covariance between values $f(s)$ and $f(t)$ over a population of curves.
We can write these bivariate functions in a \textit{bivariate basis expansion} $$ r(s,t) = \sum_{k=1}^K \sum_{l=1}^K b_{k,l} \phi_k(s) \psi_l(t) 
  = \bm{\phi}(s)' \bm{B} \bm{\psi}(t) $$
with a $K \times K$ coefficient matrix $B$ and two sets of basis functions $\phi_k$ and $\psi_l$ using \textit{Tensor Product Splines}
$$ B_{k,l}(s,t) = \phi_k(s) \psi_l(t).$$



\newpage

Statistical Shape Analysis is the branch of statistics concerned with modelling the geometry of objects.
Examples might be the outlines of bones and organs, a handwritten digit, or the folds of a protein.
To capture an object's geometrical information, a common approach is the use of so called \enquote{landmarks}: characteristic points on an object, which match between and within populations \parencite[see][3]{DrydenMardia2016}.
However, in recent years an alternative approach has become more popular, where objects are represented using curves.
This has the advantage of offering a more flexible representation of an object's geometry, as the analysis is not restricted to a set of discrete points.
Usually, the curves are themselves represented by functions $\beta : [0,1] \rightarrow \mathbb{R}^k$, which, for example in 2D, might describe the outlines of an object in an image. 
As each object corresponds to one function, this opens up a connection to the branch of statistics concerned with modelling whole functions: Functional Data Analysis \parencite[see e.g.][]{Ramsay2006}.

When analysing the geometry of objects, differences in location, rotation, and size are often not of interest.
Instead, the focus lies purely on their differences in  \textit{shape}, a widely adapted definition of which was established by \cite{Kendall1977} and which might be formulated in the following way:
\begin{definition}[Shape] 
    All geometrical information that remains when location, scale and rotational effects are removed from an object \parencite[see][1]{DrydenMardia2016}.
\end{definition}
\noindent The functional approach to shape analysis introduces an additional kind of invariance relating to the parametrisation $t \in [0,1]$, as only the image of $\beta$ but not its parametrisation is indacative of the object's shape.

\newpage
In statistical shape analysis one is interested in analysing the geometry of
objects, like the outlines of bones and organs, a handwritten digit, or the
folds of a protein.
\noindent This geometrical information is usually approximated, by measuring
the cordinates of a fixed set of \textit{landmarks}, which are characteristic
points on an object that match between and within populations
\parencite[see][3]{DrydenMardia2016}.
As these coordinates depend on the scale, rotation and translation of the
object at time of measurement, only the \textit{equivalence class} of the
landmark configuration modulo these transformations is indicative of its shape,
which will be defined in more detail in section \ref{theo:inv}.
A popular type of shape mean that does not depend on the rotation, scaling and
translation of the input shapes is the \textit{full Procrustes mean}.
Here, the mean is defined as the minimizer of a least squares criterion, using
a distance measure that is invariant under the mentioned transformations, which
will be discussed in section \ref{theo:proc}.

Instead of approximating the geometry of an object using landmarks, its whole
outline might be represented by an open or closed curve $\beta : [0,1]
\rightarrow \mathbb{R}^d,\, d \in \mathbb{N}$, eliminating the (often
subjective) decision of which points to consider as "characteristic".
This functional data approach introduces an additional kind of shape invariance
relating to the parametrisation $t$ of $\beta$, as only the image of
$\beta$ but not its parametrisation is indacative of the objects shape.
A functional shape mean, which is invariant with respect to re-parametrisation
of the input curves, is called an \textit{elastic} mean and its calculation is
greatly simplified by working in the \textit{square-root-velocity} (SRV)
framework as introduced in \cite{SrivastavaEtAl2011}.
Elastic shape means and the SRV framework will be explained in section
\ref{theo:srv}.
Even though this functional approach eliminates the subjectivity of choosing a
fixed set of landmarks, in practice, a curve will never be fully observed and
one has to again work with a set of sampled points along the curve. 
A particular challenge is working with curves that are sparsely and irregularly
sampled, where one has to employ appropriate smoothing techniques to make
maximum use of the available (sparse) data.
This will be briefly discussed in section \ref{theo:sparse}.

The aim of this thesis is to bring all these concepts together in the
estimation of \enquote{Elastic Full Procrustes Means for Sparse and Irregular
Planar Curves}. 
This is a novel approach as, firstly, the literature on elastic shape means has
so far mostly focused on the different mean concept of the Riemannian center of
mass mean \parencite{SrivastavaEtAl2011} and secondly, shape means of
sparse and irregular shape data have so far only been considered modulo
re-parametrisation and translation, excluding rotation and scaling invariance
\parencite{Steyer2021}.
The thesis focuses on the special case of 2D (i.e. planar) curves, as it can be
shown that the Procrustes mean has particularly nice properties in this
setting when working with complex notation.
Additionally this thesis will be mainly concerned with the mean estimation for
open curves, as the closed curves case is more challenging mathematically.
After covering the relevant background material  in section \ref{sec:theo}, an
expression for the elastic full Procrustes mean will be derived in section
\ref{sec:mean}.
In section \ref{sec:algo}, an estimation strategy for the setting of sparse and
irregular curves is proposed, which will be applied to simulated and empirical
datasets in section \ref{sec:app}.
Finally, in section \ref{sec:outl} a possible extension to closed curves will
be briefly discussed.
All results will be summarized in section \ref{sec:sum}.

\newpage

\label{sec:5}
In this thesis a method for shape mean estimation of open planar curves was
proposed.
The estimation method is especially suitable for sparse and irregular
observations as it makes use of covariance smoothing methods.

The method and its implementation were validated over toy datasets, where it was
shown that the estimated shape means are elastic and invariant with respect to
shape preserving transformations of the input curves.
The penalty used in the estimation of the Hermitian covariance
surface provides stable estimates in regions where observations are sparse,
where good results were achieved for first and second order penalties.
The estimated mean may be plotted together with the elastic full Procrustes
fits, however, care has to be taken when interpreting these plots, as the
estimated mean curve is not a simple functional mean of the aligned curves, due to
their shrinkage.
However, for this same reason the elastic full Procrustes mean is quite robust to
outliers, as they tend to get shrunk towards zero, restricting their influence
on the mean estimation.
Finally, the method was used to analyse variability in an empirical dataset of
tongue contours during consonant articualtion.
Here it was found that the vowel context has the largest influence on the tongues
shape, while the effect of the spoken consonant seems to vary between speakers.

The proposed methods might be improved by including some form of smoothing into
the estimation of the elastic full Procrustes fits, such as a variation of the
curve smoothing proposed in \cref{app:a-smooth}, or by directly solving the
warping optimisation over the analytical solution to the rotation and scaling
alignment.
Two important extension that seem quite challanging would be elastic full Procrustes mean estimation for
closed curves and for higher-dimensional curves.


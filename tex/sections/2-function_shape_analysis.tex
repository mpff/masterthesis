Before beginning with a derivation of the elastic full Procrustes mean, it is
important to establish a notational and mathematical framework for the
treatment of planar shapes.
While the restriction to the 2D case might seem a major one, it still covers
all shape data extracted from e.g.\ imagery and is therefore very applicable in
practice.
Examples of objects that can be analyzed in this way are handwritten digits or
symbols, the outlines bones and organs in medical images, or even movement
trajectories on a map.

All these objects may be naturally represented as planar curves $\beta : [0,1]
\rightarrow \mathbb{R}^2$ with $\beta(t) = (x(t),\, y(t))^T$, where $x(t)$ and
$y(t)$ are the scalar-valud \textit{coordinate functions}.
Calculations in 2D, and in particular the derivation of the
full Procrustes mean, are greatly simplified by using complex notation.
Going forward, we will therefore identify $\mathbb{R}^2$ with $\mathbb{C}$ and
always use complex notation when representing a planar curve:
$$\beta : [0,1] \rightarrow \mathbb{C}, \quad \beta(t) = x(t) + i\, y(t).$$

\subsection{Shape Invariance and Equivalence Classes}
\label{theo:inv}
The concept of shape is closely related to the concept of invariance under the
transformations of scaling, translation and rotation.
When considering the shape of a curve, we additionally have to take into
account invariance with respect to re-parametrisation.
This can be seen by noting that the curves $\beta(t)$ and $\beta(\gamma(t))$,
with some \textit{warping function} $\gamma : [0,1] \rightarrow [0,1]$
monotonically increasing and differentiable, have the same image and therefore
represent the same geometrical object.
We can say that the actions of translation, scaling, rotation and
re-parametrization are \textit{equivalence relations} with respect to shape, as
each action leavs the shape of the curve untouched and only changes the way the
shape is represented.
The shape of a curve can then be defined as the respective \textit{equivalence
class}, i.e. the set of all possible shape preserving transformations of the
curve.
As two equivalence classses are neccessarily either disjoint or identical, we
can consider two curves as having the same shape, if they are elements of the
same equivalence class \parencite[see][40]{SrivastavaKlassen2016}.

\begin{definition}[Equivalence relation, equivalence class and quotient space] 
    A relation $\backsim$ on a set $X$ is called an \textbf{equivalence
    relation} if, for all $x,y,z \in X$, it has the following properties:
    \begin{itemize}[noitemsep,topsep=0pt]
        \item[i.] $x \backsim x$ (reflexivity)
        \item[ii.] $x \backsim y \Rightarrow y \backsim x$ (symmetry)
        \item[iii.] $x \backsim y, y \backsim z \Rightarrow x \backsim z$
        (transitivity)
    \end{itemize}
    The \textbf{equivalence class} $[x]$ of $x \in X$ is given by the set of
    all $y \in X$ so that $x \backsim y$.
    The \textbf{quotient space} $X \big/ {\backsim}$ of $X$ under the relation
    $\backsim$ is the disjoint set of all equivalence classes in $X$. 
\end{definition}

We now want to define an equivalence relation with respect to shape.
With this in mind, let us first consider how the discussed transformations act
individually on the set of parametrized planar curves with complex
representation $\beta : [0,1]
\rightarrow \mathbb{C}$:
\begin{itemize}
  \item
    The \textbf{translation} group $\mathbb{C}$ acts on $\beta$ by $(\xi, \beta)
    \mapsto \beta + \xi$, for any $\xi \in \mathbb{C}$.
    We can consider two curves as equivalent with respect to translation
    $\beta_1 \backsim \beta_2$, if there exists a complex scalar $\tilde\xi \in
    \mathbb{C}$ so that $\beta_1 = \beta_2  + \tilde\xi$.
    For this relation an equivalence class is $[\beta] = \{\beta + \xi\, |\,
    \beta : [0,1] \rightarrow \mathbb{C},\, \xi \in \mathbb{C}\}$.
  \item 
    The \textbf{scaling} group $\mathbb{R}^+$ acts on $\beta$ by $(\lambda, \beta)
    \mapsto \lambda \beta$, for any $\lambda \in \mathbb{R}^+$.
    We define $\beta_1 \backsim \beta_2$, if there exists a scalar
    $\tilde\lambda \in \mathbb{R}^+$ so that $\beta_1 = \tilde\lambda \beta_2$.
    An equivalence class is $[\beta] = \{\lambda\beta\,|\,\beta : [0,1]
    \rightarrow \mathbb{C},\, \lambda \in \mathbb{R}^+\}$.
  \item 
    The \textbf{rotation} group $[0,2\pi]$ acts on $\beta$ by $(\theta, \beta)
    \mapsto e^{i\theta} \beta$, for any $\theta \in [0,2\pi]$.
    We define $\beta_1 \backsim \beta_2$, if there exists a $\tilde\theta \in
    [0,2\pi]$ with $\beta_1 = e^{i\tilde\theta} \beta_2$.
    An equivalence class is $[\beta] = \{e^{i\theta}\beta\,|\, \beta : [0,1]
    \rightarrow \mathbb{C},\, \theta \in [0,2\pi]\}$.
  \item 
    The \textbf{re-parametrization} group $\Gamma$ acts on $\beta$ by $(\gamma,\beta)
    \mapsto \beta \circ \gamma$, for any $\gamma \in \Gamma$ with $\Gamma$
    being the set of monotonically increasing and differentiable warping
    functions.
    We define $\beta_1 \backsim \beta_2$, if there exists a warping function
    $\tilde\gamma \in \Gamma$ with $\beta_1 = \beta_2 \circ \tilde\gamma$.
    An equivalence class is $[\beta] = \{\beta \circ \gamma\,|\, \beta : [0,1]
    \rightarrow \mathbb{C},\, \gamma \in \Gamma\}$.
\end{itemize}
In a next step, we can consider how these transformations act in concert and
whether they \textit{commute}, that is, whether the order of applying the
transformations changes outcomes.
Consider for example the actions of the rotation and scaling product group
$\mathbb{R}^+ \times [0,2\pi]$ given by $((\lambda, \theta), \beta) \mapsto
\lambda e^{i\theta} \beta$, which clearly commute as $\lambda
(e^{i\theta}\beta) = e^{i\theta}(\lambda\beta)$.
However, the joint actions of scaling, rotation, and translation do not
commute, as $\lambda e^{i\theta}(\beta + \xi) \neq \lambda e^{i\theta}\beta +
\xi$ and therefore, the order of translating and rotating or scaling matters.
\textit{Euclidean similarity transformations}.
\begin{definition}[Euclidean similarity transformation] 
  We define an \textbf{Euclidean similarity transformation} on a curve $\beta :
  [0,1] \rightarrow \mathbb{C}$ as the joint action of scaling, rotation, and
  translation by $((\xi, \lambda, \theta), \beta) \mapsto \lambda e^{i\theta}
  \beta + \xi$, with $\xi \in \mathbb{C}$, $\lambda \in \mathbb{R}^+$, and
  $\theta \in [0,2\pi]$ \parencite[see][62]{DrydenMardia2016}.
\end{definition}
With respect to the action of re-parametrization, we can note that it
necessarily commutes with all Euclidean similarity transformations as those
only change the image of $\beta$, while the former only changes the
parametrization.
Putting everything together, we can finally give a mathematical definition of
the shape of a planar curve.
\begin{definition}[Shape]
  The \textbf{shape} of a planar curve $\beta : [0,1] \rightarrow \mathbb{C}$
  is given by the equivalence class with respect to all Euclidean similarity
  transformations and re-parametrizations of $\beta$
  $$ [\beta] = \{\lambda e^{i\theta}(\beta \circ \gamma) + \xi\,|\, \xi \in
  \mathbb{C},\, \lambda \in \mathbb{R}^+,\, \theta \in [0,2\pi],\, \gamma \in
  \Gamma\} $$
\end{definition}





\subsection{The Full Procrustes Mean for Planar Curves}
\label{theo:proc}
\begin{definition}[Full Procrustes distance, full Procrustes mean]
    For $X_1, X_2$ landmark configurations, represented as $m \times d$
    matrices with $m$ landmarks in $d$ dimensions, the \textbf{full Procrustes
    distance} between their shapes $[X_1], [X_2]$ is defined as
    $$d_F([X_1], [X_2]) = 
      \inf_{\lambda \in \mathbb{R}_+,\, \Gamma \in SO(m)} ||\widetilde{X}_1 - \lambda
      \widetilde{X}_2\Gamma||, $$
    where $\widetilde{X}_{1,2}$ are centered and normalized landmark
    configurations, $\lambda \in \mathbb{R}_+$ is a scaling factor and $\Gamma
    \in SO(d)$ a rotation matrix.

    The \textbf{full Procrustes mean} shape for a sample of landmark
    configurations $X_i$ ($i = 1,\dots,n$) is then given by the equivalence
    class [$\hat\mu_F$] of a landmark configuration that minimizes the sum of
    squared full Procrustes distances
    $$\hat{\mu}_F = \arginf_{\mu} \sum_{i=1}^n d_F([\mu], [X_i])^2, $$
    where $\mu$ is assumed centered and normalized
    \parencites[see][71,114]{DrydenMardia2016}.
\end{definition}


\subsection{Elastic Means and the Square-Root-Velocity Framework}
\label{theo:srv}


\subsection{Functional Data Analysis of Sparse and Irregular Planar Curves}
\label{theo:sparse}


\newpage

Before beginning with a derivation of the elastic full Procrustes mean, it is
important to establish a notational and mathematical framework for the
treatment of planar shapes.
While the restriction to the 2D case might seem a major one, it still covers
all shape data extracted from e.g.\ imagery and is therefore very applicable in
practice.
Examples of objects that can be analyzed in this way are handwritten digits or
symbols, the outlines bones and organs in medical images, or even movement
trajectories on a map.

All these objects may be naturally represented as planar curves $\beta : [0,1]
\rightarrow \mathbb{R}^2$ with $\beta(t) = (x(t),\, y(t))^T$, where $x(t)$ and
$y(t)$ are the scalar-valud \textit{coordinate functions}.
Calculations in 2D, and in particular the derivation of the
full Procrustes mean, are greatly simplified by using complex notation.
Going forward, we will therefore identify $\mathbb{R}^2$ with $\mathbb{C}$ and
always use complex notation when representing a planar curve:
$$\beta : [0,1] \rightarrow \mathbb{C}, \quad \beta(t) = x(t) + i\, y(t).$$

\subsection{Equivalence Classes and Shape Invariance}
\label{theo:inv}
The concept of shape is closely related to the concept of invariance under the
transformations of scaling, translation and rotation.
When considering the shape of a curve, we additionally have to take into
account invariance with respect to re-parametrisation.
This can be seen by noting that the curves $\beta(t)$ and $\beta(\gamma(t))$,
with some \textit{warping function} $\gamma : [0,1] \rightarrow [0,1]$
monotonically increasing and differentiable, have the same image and therefore
represent the same geometrical object.
We can say that the actions of translation, scaling, rotation and
re-parametrization are \textit{equivalence relations} with respect to shape, as
each action leaves the shape of the curve untouched and only changes the way
the shape is represented.

The shape of a curve can then be defined as the respective \textit{equivalence
class}, i.e. the set of all possible shape preserving transformations of the
curve.
As two equivalence classses are neccessarily either disjoint or identical, we
can consider two curves as having the same shape, if they are elements of the
same equivalence class \parencite[see][40]{SrivastavaKlassen2016}.

\begin{definition}[Equivalence relation, equivalence class and quotient space] 
    A relation $\backsim$ on a set $X$ is called an \textbf{equivalence
    relation} if, for all $x,y,z \in X$, it has the following properties:
    \begin{itemize}[noitemsep,topsep=0pt]
        \item[i.] $x \backsim x$ (reflexivity)
        \item[ii.] $x \backsim y \Rightarrow y \backsim x$ (symmetry)
        \item[iii.] $x \backsim y, y \backsim z \Rightarrow x \backsim z$
        (transitivity)
    \end{itemize}
    The \textbf{equivalence class} $[x]$ of $x \in X$ is given by the set of
    all $y \in X$ so that $x \backsim y$.
    The \textbf{quotient space} $X \big/ {\backsim}$ of $X$ under the relation
    $\backsim$ is the set of all equivalence classes in $X$. 
\end{definition}

\noindent We now want to define an equivalence relations with respect to shape.
With this in mind, let us first consider how the discussed transformations act
individually on the set of parametrized planar curves with complex
representation $\beta : [0,1]
\rightarrow \mathbb{C}$:
\begin{itemize}
  \item
    The \textbf{translation} group $\mathbb{C}$ acts on $\beta$ by $(\xi, \beta)
    \mapsto \beta + \xi$, for any $\xi \in \mathbb{C}$.
    We can consider two curves as equivalent with respect to translation
    $\beta_1 \backsim \beta_2$, if there exists a complex scalar $\tilde\xi \in
    \mathbb{C}$ so that $\beta_1 = \beta_2  + \tilde\xi$.
    For this relation an equivalence class is $[\beta] = \{\beta + \xi\, |\,
    \beta : [0,1] \rightarrow \mathbb{C},\, \xi \in \mathbb{C}\}$.
  \item 
    The \textbf{scaling} group $\mathbb{R}^+$ acts on $\beta$ by $(\lambda, \beta)
    \mapsto \lambda \beta$, for any $\lambda \in \mathbb{R}^+$.
    We define $\beta_1 \backsim \beta_2$, if there exists a scalar
    $\tilde\lambda \in \mathbb{R}^+$ so that $\beta_1 = \tilde\lambda \beta_2$.
    An equivalence class is $[\beta] = \{\lambda\beta\,|\,\beta : [0,1]
    \rightarrow \mathbb{C},\, \lambda \in \mathbb{R}^+\}$.
  \item 
    The \textbf{rotation} group $[0,2\pi]$ acts on $\beta$ by $(\theta, \beta)
    \mapsto e^{i\theta} \beta$, for any $\theta \in [0,2\pi]$.
    We define $\beta_1 \backsim \beta_2$, if there exists a $\tilde\theta \in
    [0,2\pi]$ with $\beta_1 = e^{i\tilde\theta} \beta_2$.
    An equivalence class is $[\beta] = \{e^{i\theta}\beta\,|\, \beta : [0,1]
    \rightarrow \mathbb{C},\, \theta \in [0,2\pi]\}$.
  \item 
    The \textbf{re-parametrization} group $\Gamma$ acts on $\beta$ by $(\gamma,\beta)
    \mapsto \beta \circ \gamma$, for any $\gamma \in \Gamma$ with $\Gamma$
    being the set of monotonically increasing and differentiable warping
    functions.
    We define $\beta_1 \backsim \beta_2$, if there exists a warping function
    $\tilde\gamma \in \Gamma$ with $\beta_1 = \beta_2 \circ \tilde\gamma$.
    An equivalence class is $[\beta] = \{\beta \circ \gamma\,|\, \beta : [0,1]
    \rightarrow \mathbb{C},\, \gamma \in \Gamma\}$.
\end{itemize}
In a next step, we can consider how these transformations act in concert and
whether they \textit{commute}, that is, whether the order of applying the
transformations changes outcomes.
Consider for example the actions of the rotation and scaling product group
$\mathbb{R}^+ \times [0,2\pi]$ given by $((\lambda, \theta), \beta) \mapsto
\lambda e^{i\theta} \beta$, which clearly commute as $\lambda
(e^{i\theta}\beta) = e^{i\theta}(\lambda\beta)$.
However, the joint actions of scaling, rotation, and translation do not
commute, as $\lambda e^{i\theta}(\beta + \xi) \neq \lambda e^{i\theta}\beta +
\xi$.
It follows that the order of translating and rotating or scaling matters and
when defining such a joint action, one usually takes the translation to act on
the already scaled and rotated curve.
\begin{definition}[Euclidean similarity transformation] 
  We define an \textbf{Euclidean similarity transformation} of a curve $\beta :
  [0,1] \rightarrow \mathbb{C}$ as the joint action of scaling, rotation, and
  translation by 
  $$((\xi, \lambda, \theta), \beta) \mapsto \lambda e^{i\theta} \beta + \xi,$$
  with $\xi \in \mathbb{C}$, $\lambda \in \mathbb{R}^+$, and $\theta \in
  [0,2\pi]$ \parencite[see][62]{DrydenMardia2016}.
\end{definition}
\noindent With respect to the action of re-parametrization, we can note that it
necessarily commutes with all Euclidean similarity transformations, as those
only act on the image of $\beta$, while the former only acts on the
parametrization.
Putting everything together, we can finally give a formal definition of the
shape of a planar curve.
\begin{definition}[Shape]
  The \textbf{shape} of a planar curve $\beta : [0,1] \rightarrow \mathbb{C}$
  is given by its equivalence class with respect to all Euclidean similarity
  transformations and re-parametrizations
  $$ [\beta] = \left\{\lambda e^{i\theta}(\beta \circ \gamma) + \xi\,|\, \xi
  \in \mathbb{C},\, \lambda \in \mathbb{R}^+,\, \theta \in [0,2\pi],\, \gamma
  \in \Gamma\right\}. $$
  The \textbf{shape space} is then given by the corresponding quotient space 
  $$\mathbb{C}^{[0,1]} \big/ \mathbb{C} \rtimes \left( \mathbb{R}^+ \times
  [0,2\pi] \right) \times \Gamma = \left\{[\beta]|\beta \in
  \mathbb{C}^{[0,1]}\right\},$$
  where the symbol \enquote{$\rtimes$} denotes a semi-direct product, i.e. that
  the translation group acts \enquote{after} scaling and rotation
  \parencite[for details see][Chapter 3]{SrivastavaKlassen2016}.
\end{definition}

\noindent As we are interested in calculating shape means, we now need to find a way to
measure simmilarity and dissimilarity between shapes.
More formally, this means that we want to measure distances between equivalence
classes in shape space.
As equivalence classes and quotient spaces are already quite abstract concepts
and as the discussed spaces may be quite complex, we need to find a way of
simplifying calculations in shape space.
The basic idea is to find unique curves as a representation for each equivalence
class, so that distance in shape space can be equivalently calculated as the
distance between these curves.
In short we want to find intersections of the function space, so that there is
a one-to-one relationship between elements of that interesection and the
squivalence classes in shape space.
Consider, for example, the action of the translation group $\mathbb{C}$
with corresponding quotient space $\mathbb{C}^{[0,1]} \big/ \mathbb{C} =
\left\{[\beta] \,|\, \beta \in \mathbb{C}^{[0,1]} \right\}$ and $[\beta] =
\{\beta + \xi\,|\, \xi \in \mathbb{C}\}$.
For each equivalence class in $\mathbb{C}^{[0,1]} \big/ \mathbb{C}$ there is
exactly one element in the space of all centered curves $\mathcal{C}_1 =
\left\{ \beta \in \mathbb{C}^{[0,1]}\,\big|\, \int_0^1 \beta(t) \, dt = 0
\right\}$ belonging to that equivalence class.
For the scaling group $\mathbb{R}^+$, when considering only continuously
differentiable $\beta$, we can similarly form the space of all unit-length
curves $\mathcal{C}_2 = \left\{\beta \in \mathbb{C}^{[0,1]}\,\big|\,
\beta\,\text{continuously differentiable},\, \int_0^1 |\dot{\beta}(t)| \, dt =
1 \right\}$ and for each equivalence class with respect to scaling there is
again exactly one element in $\mathcal{C}_2$ belonging to that class.
Both intersections taken together result in the concept of \textit{pre-shape}
and the act of removing translation and scaling variability by centering and
restricting curves to unit-length is called \enquote{quotienting out} \parencite[see][133\,f.]{SrivastavaKlassen2016}.
\begin{definition}[Pre-shape]
  The \textbf{pre-shape} of a countinously differentiable planar curve $\beta :
  [0,1] \rightarrow \mathbb{C}$ is given by its centered unit-length
  representation 
  $$\tilde\beta = l_\beta^{-1} \left(\beta - c_\beta\right)$$
  with $l_\beta = \int_0^1|\dot\beta(t)|\,dt$ and $c_\beta = \int_0^1
  \beta(t)\,dt$.
  The \textbf{shape} of $\beta$ can then also be defined as
  $$[\beta] = \left\{e^{i\theta}(\tilde\beta \circ \gamma) \,\big|\, \theta \in
  [0,2\pi],\, \gamma \in \Gamma\right\}.$$
\end{definition}
\noindent We can use the concept of pre-shape to construct an appropriate distance
measure for equivalence classes with respect to scaling and translation, by
using the usual distance metric for complex functions with the respective
pre-shapes.
$$d_{\mathbb{C}^{[0,1]} \big/ \mathbb{C} \rtimes
\mathbb{R}^+}([\beta_1], [\beta_2]) = d_{\mathbb{C}^{[0,1]}}(\tilde\beta_1,
\tilde\beta_2)$$
\noindent Working with pre-shapes means that we only have to take care of shape
equivalence with respect to rotation and re-parametrization going forward.
However, both types of transformations are not as easily dealt with as
translation and scaling.
We will now take a look at a different way of constructing distance measures
for quotient spaces, one dealing with rotation and one dealing with
re-parametrization.


\subsection{The Full Procrustes Mean for Planar Curves}
\label{theo:proc}

The \textit{full Procrustes distance} is a widely used distance function in
classical shape analysis that allows for scale and rotation invariant shape
distance calculation.
To remove translation and scaling variablity, we chose to restrict our curves
to a certain fixed position and scale.
This works, because we can simply calculate both for any curve.
However, there is no similar way to calculate the \enquote{rotation} of a
curve and in fact, saying that a curve is \enquote{rotated} is only meaningful in
comparison to other curves. 
A solution is to align the rotation of any two curves on a case by case basis,
when calculating the distance between them.
\begin{definition}[Full Procrustes distance]
  The \textbf{full Procrustes distance} between the shapes $[\beta_1]$,
  $[\beta_2]$ of two continously differentiable $\beta_1$, $\beta_2$ with
  $\beta_i : [0,1] \rightarrow \mathbb{C}$ is given by 
    $$d_F([\beta_1], [\beta_2]) = \inf_{\lambda \in \mathbb{R}^+,\, \theta \in
    [0,2\pi]} ||\tilde{\beta}_1 - \lambda e^{i\theta} \tilde{\beta}_2||, $$
    where $\tilde\beta_{1,2}$ are the centered, unit-length pre-shapes of
    $\beta_{1,2}$.
\end{definition}
\begin{definition}[Full Procrustes mean]
    The \textbf{full Procrustes mean} shape for a sample of landmark
    configurations $X_i$ ($i = 1,\dots,n$) is then given by the equivalence
    class [$\hat\mu_F$] of a landmark configuration that minimizes the sum of
    squared full Procrustes distances
    $$\hat{\mu}_F = \arginf_{\mu} \sum_{i=1}^n d_F([\mu], [X_i])^2, $$
    where $\mu$ is assumed centered and normalized
    \parencites[see][71,114]{DrydenMardia2016}.
\end{definition}

\textbf{[Define an intersection of the quotient space by projections. Proof that this is good. Procrustes Fits.]}

\subsection{Elastic Means and the Square-Root-Velocity Framework}
\label{theo:srv}
Instead of working with the original curve $\beta$, for calculation of
\textit{elastic} means it is advantageous to work with its corresponding
\textit{square root velocity curve} given by
$$ q : [0, 1] \rightarrow \mathbb{C}, \quad q(t) =
\frac{\dot{\beta}(t)}{\sqrt{|| \dot{\beta}(t) ||}} \quad \text{for} \,\,
\dot{\beta}(t) \neq 0 $$
where original curve $\beta$ can be obtained up to translation by back
transformation via $\beta(t) = \beta(0) + \int_0^t q(s) || q(s) || ds$.
Moreover, if the original curve is of unit length the SRV curve will be
automatically normalized:
$$ ||q|| = \sqrt{\langle q, q \rangle} = \sqrt{ \int_0^1 \overline{q(t)} q(t)
\, dt } = \sqrt{ \int_0^1 |q(t)|^2 \, dt} = \sqrt{\int_0^1 |\dot{\beta}(t)| \,
dt} = \sqrt{1} = 1. $$
\subsection{Functional Data Analysis of Sparse and Irregular Planar Curves}
\label{theo:sparse}


\newpage

\subsection{Shape Invariance and Equivalence Classes}
\label{intro:inv}
Before beginning with further derivations and definitions it is important to establish a mathematical framework for treating shapes.

\begin{definition}[Equivalence relation, equivalence class and quotient space] 
    A relation $\backsim$ on a set $X$ is called an \textbf{equivalence
    relation} if, for all $x,y,z \in X$, it has the following properties:
    \begin{itemize}[noitemsep,topsep=0pt]
        \item[i.] $x \backsim x$ (reflexivity)
        \item[ii.] $x \backsim y \Rightarrow y \backsim x$ (symmetry)
        \item[iii.] $x \backsim y, y \backsim z \Rightarrow x \backsim z$
        (transitivity)
    \end{itemize}
    The \textbf{equivalence class} $[x]$ of $x \in X$ is given by the set of
    all $y \in X$ so that $x \backsim y$.
    The \textbf{quotient space} $X \big/ {\backsim}$ of $X$ under the relation
    $\backsim$ is the disjoint set of all equivalence classes in $X$ 
    \parencite[see][40]{SrivastavaKlassen2016}.
\end{definition}
\noindent This means, that any statistical analysis of shapes should ultimately
work with these equivalence classes as well.


\subsection{The Full Procrustes Mean in Classical Shape Analysis}
\label{intro:proc}
\begin{definition}[Full Procrustes distance, full Procrustes mean]
    For $X_1, X_2$ landmark configurations, represented as $m \times d$
    matrices with $m$ landmarks in $d$ dimensions, the \textbf{full Procrustes
    distance} between their shapes $[X_1], [X_2]$ is defined as
    $$d_F([X_1], [X_2]) = 
      \inf_{\lambda \in \mathbb{R}_+,\, \Gamma \in SO(m)} ||\widetilde{X}_1 - \lambda
      \widetilde{X}_2\Gamma||, $$
    where $\widetilde{X}_{1,2}$ are centered and normalized landmark
    configurations, $\lambda \in \mathbb{R}_+$ is a scaling factor and $\Gamma
    \in SO(d)$ a rotation matrix.

    The \textbf{full Procrustes mean} shape for a sample of landmark
    configurations $X_i$ ($i = 1,\dots,n$) is then given by the equivalence
    class [$\hat\mu_F$] of a landmark configuration that minimizes the sum of
    squared full Procrustes distances
    $$\hat{\mu}_F = \arginf_{\mu} \sum_{i=1}^n d_F([\mu], [X_i])^2, $$
    where $\mu$ is assumed centered and normalized
    \parencites[see][71,114]{DrydenMardia2016}.
\end{definition}


\subsection{Elastic Means and the Square-Root-Velocity Framework}
\label{intro:srv}
\textbf{[How is functional shape data analyzed]}


\subsection{Functional Data Analysis of Sparse and Irregular Planar Curves}
\label{intro:sparse}
\textbf{[What is sparse and irregular data]}



\newpage

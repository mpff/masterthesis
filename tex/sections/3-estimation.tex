\textbf{[TODO: Explain the problem!]}
Algorithm \ref{algo:mean} shows an idealized version of the elastic full Procrustes mean estimation, where it is assumed that each curve $\beta_i$ is fully observed.
This is not the case in practice, as each observation $\beta_i$ is usually itself only observed at a finite number of discrete points $\beta_i(t_{i1}), \dots, \beta_i(t_{im_i})$.
Additionally, the number of observed points per curve $m_i$ might be quite small and the points do not need to follow a common sampling scheme across all curves, a setting which is respectively known as \emph{sparse} and \emph{irregular}.
In this section an estimation strategy for dealing with sparse and irregular observations is proposed.

\section{Sparse Treatment of SRV curves}
A natural first consideration might be how to calculate a SRV curve from sparse observations.
Following \cite{Steyer2021} we may treat a discretely observed curve $\beta$ as piecewise linear between its observed corners. 
As usually only the image $\beta(t_1),\dots,\beta(t_m)$ but not the parametrisation $t_1,\dots,t_m$ is observed, it is necessary to construct an initial parametrsiation.
A natural choice might be an \emph{arc-length-parametrisation}, where we set $t_j = \frac{l_j}{l}$ with $l_j = \sum_{k=1}^{j-1} \abs{\beta(t_{k+1}) - \beta(t_k)}$ the polygon-length up to point $j$ for $j \leq 2$, $l_1 = 0$ and $l_m = l$.
Considering the discrete derivative $\Delta \beta \bigg\rvert_{[t_j, t_{j+1}]} = \frac{\beta(t_{j+1}) - \beta(t_{j})}{t_{j+1} - t_j}$i, the corresponding SRV curve $q$ can then be treated as piecewise constant $q\bigg\rvert_{[t_{j},t_{j+1}]} = q_j$ with $q_j = \Delta \beta \bigg\rvert_{[t_j, t_{j+1}]} \middle/ \sqrt{\norm{\Delta \beta \bigg\rvert_{[t_j, t_{j+1}]}}} $ the discrete \emph{square-root-velocity} of $\beta$ between the corners $\beta(t_j)$ and $\beta(t_{j+1})$, $j=1,\dots,m-1$.


\section{Estimating the Full Procrustes Mean in a Fixed Basis}
\textbf{[TODO: Explain smoothing!]} 


\textbf{[TODO: Motivate b-spline basis! Lisa?]}

\textbf{[TODO: Derivation in a fixed basis]}

\begin{comment}
To avoid having to sample the estimated covariance surface $\hat{C}(s,t)$ on a large grid when calculating its leading eigenfunction, it might be preferable to calculate this eigenfunction from the vector of basis coefficients directly.
After choosing a basis representation $b = (b_1, \dots, b_k)$ with $b_j : \mathbb{R} \rightarrow \mathbb{R}$ real-valued basis functions, we want to estimate complex coefficients $\theta_j \in \mathbb{C}$ so that the Full Procrustes mean of SRV curves is given by $\hat{\mu}(t) = \sum_{j=1}^k \hat{\theta}_j b_j(t) = b^T \hat{\theta}$:
\begin{align*}
    \hat{\mu} =& \argmax_{\theta : ||b^T\theta||=1} \sum_{i=1}^n \langle b^T\theta, q_i \rangle \langle q_i, b^T\theta \rangle \\
    =& \argmax_{\theta : ||b^T\theta||=1} \sum_{k,l} \sum_{i=1}^n \langle b_k \theta_k, q_i \rangle \langle q_i, b_l \theta_l \rangle \\
    =& \argmax_{\theta : ||b^T\theta||=1} \sum_{k,l} \bar{\theta}_k \theta_l \sum_{i=1}^n \langle b_k, q_i \rangle \langle q_i, b_l \rangle \\
    =& \argmax_{\theta : ||b^T\theta||=1} \theta^H S \theta \\
\end{align*}
where the matrix $S = \left\{ \sum_{i=1}^n \langle b_k, q_i \rangle \langle q_i, b_l \rangle \right\}_{k,l}$ has to be estimated from the observed SRV curves.
We can further simplify $S$ to
\begin{align*}
    S_{kl} =& \sum_{i=1}^n \int_0^1 \bar{b}_k(t) q_i(t) dt \int_0^1 \bar{q}_i(s) b_l(s) ds \\
    =& \int_0^1 \int_0^1 \bar{b}_k(t) \underbrace{\left( \sum_{i=1}^n q_i(t) \bar{q}_i(s) \right)}_{= n \, \hat{C}(s,t)} b_l(s) ds dt\\
    =& n \, \int_0^1 \int_0^1 \bar{b}_k(t) \hat{C}(s,t) b_l(s) ds dt\\
\end{align*}
with $\hat{C}(s,t) = \frac{1}{n} \sum_{i=1}^n q_i(s) \overline{q_i(t)}$ the sample analogue to the complex population covariance function $C(s,t) = \mathbb{E}[q(s)\overline{q(t)}]$.
We may estimate $C(s,t)$ via tensor product splines, so that $\hat{C}(s,t) = \sum_{k,l} \hat{\xi}_{kl} b_k(t) b_l(s)$, where $b_j(t)$, $j=1,\dots,k$ are the same real valued basis functions as used for the mean and $\hat{\xi}_{kl}$ are the estimated complex coefficients.
We can then further simplify $S_{kl}$
\begin{align*}
    S_{kl} =& n \, \int_0^1 \int_0^1 b_k(t) \left( \sum_{p,q} \hat{\xi}_{pq} b_q(t) b_p(s) \right) b_l(s) ds dt\\
    =& n \, \sum_{p,q} \hat{\xi}_{pq} \int_0^1 \int_0^1 b_k(t) b_q(t) b_p(s) b_l(s) ds dt\\
    =& n \, \sum_{p,q} \hat{\xi}_{pq} \langle b_k, b_q \rangle \langle b_p, b_l \rangle\\
    =& n \, \sum_{p,q} \hat{\xi}_{pq} g_{kq} g_{pl}
\end{align*}
where $g_{ij}$, $i,j = 1, \dots, k$ are the elements of the Gram matrix $G = bb^T$ with $G = \mathbb{I}_k$ in the special case of an orthogonal basis.
We can then write the write the matrix $S$ as a function of the estimated coefficient matrix $\hat{\Xi} = (\hat{\xi}_{ij})_{i,j = 1, \dots, k}$ :
\begin{align*}
    S =& n \, G \hat{\Xi} G
\end{align*}
The full Procrustes mean of SRV curves is then given by the solution to the optimization problem
\begin{align*}
    \hat{\mu} =& \argmax_{\theta} n \, \theta^H G \hat{\Xi} G \theta \quad \text{subj. to} \quad ||b^T \theta|| = 1 \\
    =& \argmax_{\theta : ||b^T\theta||=1} \, \theta^H G \hat{\Xi} G \theta \quad \text{subj. to} \quad \theta^H G \theta = 1
\end{align*}
One may solve this by using Lagrange optimization with the Langrangian
\begin{align*}
  \mathcal{L}(\theta,\lambda) =& \, \theta^H G \hat{\Xi} G \theta - \lambda ( \theta^H G \theta - 1)
\end{align*}

\end{comment}



\section{Efficient Estimation using Hermitian Covariance Smoothing}

\textbf{[TODO: Explain estimator, GLM, hermitioan!]}

\textbf{[Fast symmetric additive cov smoothing, skew-symmetry, population vs. sample, etc.]}

\textbf{[TODO: Distances vs Means! Compound estimators vs singleton estimates.]}



\begin{comment}
Consider the following model for independent curves
\begin{equation}
    Y_i(t_{ij}) = \mu(t_{ij}, \mathbf{x}_i) + E_i(t_{ij}) + \epsilon(t_{ij}),
    \quad j = 1,\dots,D_i, \, i = 1,\dots,n,
\end{equation}
\end{comment}


\section{Numerical Integration of the Procrustes Fits}

\textbf{[TODO: Derivation as with Lisa.]}


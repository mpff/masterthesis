Alg. \ref{algo:mean} shows an idealized version of the elastic full Procrustes mean estimation, where it is assumed that each curve $\beta_i$ is fully observed.
This is not the case in practice, as each observation $\beta_i$ is usually itself only observed at a finite number of discrete points $\beta_i(t_{i1}), \dots, \beta_i(t_{im_i})$.
Additionally, the number of observed points per curve $m_i$ might be quite small and the points do not need to follow a common sampling scheme across all curves, a setting which is respectively known as \emph{sparse} and \emph{irregular}.

Following the steps laid out in Alg. \ref{algo:mean}, this section proposes a mean estimation strategy for dealing with sparse and irregular observations.
In a first step, the construction of SRV and warped SRV curves from discrete (and possibly sparse) observations will be shown in Section \ref{sec:discrete}.
Section \ref{sec:cov} discusses efficient estimation of the complex covariance surface $C(s,t)$ from sparse observations.
In Section \ref{sec:mean_basis} the calculation of the leading eigenfunction $u_1$ of $C(s,t)$ in a fixed basis will be shown.
Section \ref{sec:pfit} deals with the estimation of the scalar product $\omega = \langle z, \mu_q \rangle$, which gives the optimal rotation and scaling alignment.
Note that the final warping alignment step in Alg. \ref{algo:mean} can be solved by using methods for warping alignment of sparse and irregular curves provided in \cite{Steyer2021}.


\section{Discrete Treatment of SRV Curves}
\label{sec:discrete}
A natural first consideration might be how to calculate SRV curves from sparse observations.
We defined the SRV curve of a function $\beta \in \mathcal{AC}([0,1],\,\mathbb{C})$ as $q = \frac{\dot\beta}{\sqrt{\norm{\dot\beta}}}$ (for $\dot\beta \neq 0$).
This means that if we want to calculate the SRV curve, we have to calculate the derivative of $\beta$.
As we never observe the whole function $\beta$ but only a set discrete points $\beta(t_1),\dots,\beta(t_m)$, this is already not straight forward, as we cannot simply calculate a pointwise derivative.
However, following \cite{Steyer2021}, we treat a discretely observed curve $\beta$ as piecewise linear between its observed corners $\beta(t_1),\dots,\beta(t_m)$, which allows us to calculate a piecewise constant derivative on the intervalls $[t_j,\,t_{j+1}]$, $j=1,\dots,m-1$.
Let us first consider the case of unwarped observations.

\textbf{[TODO: Eventuell den ganzen Part mit den piecewise-constant curves streichen. Am ende wird sowieso nur die Mittelwert-Approximation benutzt.]}

\paragraph{Initial parameterization}
Usually only the image $\beta(t_1),\dots,\beta(t_m)$, but not the parametrisation $t_1,\dots,t_m$, is observed.
Therefore it is first necessary to construct an initial parameterisation.
A common choice is the \emph{arc-length-parametrisation}, where we set $t_j = \frac{l_j}{l}$ with $l_j = \sum_{k=1}^{j-1} \abs{\beta(t_{k+1}) - \beta(t_k)}$ the polygon-length up to point $j$ for $j \leq 2$, $l_1 = 0$ and $l_m = l$.

\paragraph{Piecewise-constant SRV curve} 
Consider the discrete derivative $\Delta \beta \big\rvert_{[t_j, t_{j+1}]} = \frac{\beta(t_{j+1}) - \beta(t_{j})}{t_{j+1} - t_j}$, which assumes that $\beta$ is linear between its observed corners. 
The corresponding SRV curve $q$ can then be treated as piecewise constant $q\big\rvert_{[t_{j},t_{j+1}]} = q_j$ with $q_j = \Delta \beta \big\rvert_{[t_j, t_{j+1}]} \Big/ \sqrt{\norm{\Delta \beta \big\rvert_{[t_j, t_{j+1}]}}} = \frac{\beta(t_{j+1}) - \beta(t_j)}{\sqrt{t_{j+1} - t_j} \cdot \sqrt{\norm{\beta(t_{j+1}) - \beta(t_j)}}}$ the discrete \emph{square-root-velocity} of $\beta$ between the corners $\beta(t_j)$ and $\beta(t_{j+1})$.

\paragraph{Approximate discrete SRV curve} As shown in \cite{Steyer2021} (cf.\ Fig.\ 3), treating the SRV curves as piecewise-constant functions can lead to overfitting, where the mean shape is estimated too polygon-like.
As an alternative they propose to approximate the derivative, by assuming that it attains the value of the discrete derivative $\Delta \beta \big\rvert_{[t_j,t_{j+1}]}$ at the center $s_j = \frac{t_{j+1} - t_j}{2}$ of the interval $[t_j, t_{j+1}]$.
Using this, we can construct \enquote{approximate observations} $q(s_j) \approx q_j$ of the SRV curve $q$.

\paragraph{Normalization}
We can approximate the normalized SRV curve $z$ using the polygon-length $l$ of $\beta$ by $z_j = q_j \big/ \sqrt{l}$.

\paragraph{} 
\textbf{[TODO: Nochmal überlegen mit $\gamma^{-1}$ und $\gamma$. Gerade ist das hier glaube ich falsch.]}
Let us now consider a warping function $\gamma$.
The warped discrete derivative is given by $\Delta (\beta \circ \gamma) \big\rvert_{[\gamma(t_j), \gamma(t_{j+1})]} = \frac{\beta(\gamma(t_{j+1})) - \beta(\gamma(t_{j}))}{\gamma(t_{j+1}) - \gamma(t_j)}$.
The corresponding warped SRV curve is then given by  $(q \circ \gamma) \sqrt{\dot\gamma} \big\rvert_{[\gamma(t_j), \gamma(t_{j+1})]} = \frac{\beta(\gamma(t_{j+1})) - \beta(\gamma(t_j))}{\sqrt{\gamma(t_{j+1}) - \gamma(t_j)} \cdot \sqrt{\norm{\beta(\gamma(t_{j+1})) - \beta(\gamma(t_j))}}}$.
\textbf{[TODO: Das hier zuende formulieren.]}



\section{Efficient Estimation using Hermitian Covariance Smoothing}
\label{sec:cov}

We want to estimate $C(s,t) = \mathbb{E}[z(s)\overline{z(t)}]$ given approximate observations of the normalized SRV curves $z_i(s_{ij})$, with $j = 1,\dots,m_i-1$ and $i=1,\dots,N$, where $m_i$ denotes the number of observed points per curve.
Following \textbf{[TODO: Cite]}, we can treat this estimation as a smoothing problem, by constructing responses $y_{ijk} = z_i(s_{ij}) \overline{z_i(s_{ik})}$ and treating the pairs $s_{ij}$, $s_{ik}$ as covariates $s$ and $t$.
Smooothing the responses $y_{ijk}$ gives an estimate $\hat C(\cdot, \cdot)$ of $C(s,t)$, as each response has expectation $\mathbb{E}[y_{ijk}] = C(s_{ij},s_{ik})$. 
A popular approach \textbf{[TODO: Cite]} is to carry out the smoothing in a \emph{tensor product spline} basis 
$$ C(s,t) = b(s)^T \Xi b(t) $$
where $b(s) = (b_1(s),\dots,b_K(s))$ denotes the vector of a spline basis and $\Xi$ is a $K \times K$ coefficient matrix to be estimated.
As $C(s,t)$ is complex, we restrict the spline basis to be real-valued with $b_k : \mathbb{R} \rightarrow \mathbb{R}$ for $k = 1,\dots,K$ and the coefficient matrix to be complex-valued with $\Xi \in \mathbb{C}^{K \times K}$, without loss of generality.


Considering the symmetry properties of the covariance surface allows for more efficient estimation, as shown in \cite{CederbaumScheiplGreven2018} for real valued, symmetric covariance surfaces, by considering every unique pair $s_{ij}$,$s_{ik}$ only once.
In the complex case, the covariance surface is hermitian with $C(s,t) = \overline{C(t,s)}$, which means we can decompose the estimation into two seperate regression problems over the symmetric real and skew-symmetric imaginary parts of $C(s,t)$.
$$\mathbb{E}[\Re(y)] = b(s)^T \Xi_{\Re} b(t)$$
$$\mathbb{E}[\Im(y)] = b(s)^T \Xi_{\Im} b(t)$$
with $\Xi_\Re, \Xi_\Im \in \mathbb{R}^{K\times K}$ and $\Xi = \Xi_\Re + i \Xi_\Im$, under the constraints that $\Xi_\Re^T = \Xi_\Re$ and $\Xi_\Im^T = - \Xi_\Im$.
In this thesis I estimate $\Xi_\Re$ and $\Xi_\Im$ using the \texttt{R} \parencite{Rcore} package \texttt{mgcv} \parencite{Wood2017}.
For efficient estimation, two \texttt{mgcv} smooths from the package \texttt{sparseFLMM} \parencite{sparseFLMM} are used, which implement and generalize the approach proposed by \cite{CederbaumScheiplGreven2018} for symmetric and skew-symmetric tensor product p-splines.
\textbf{[TODO: Erklärung p-splines?]


\section{Estimating the Full Procrustes Mean in a Fixed Basis}
\label{sec:mean_basis}

\textbf{[TODO: Explain smoothing!]} 
Although observations might be sparse and irregular, we want to model the mean as a whole function.
In FDA this is usally achieved by estimating the mean function using a specific set of \emph{basis functions} $b_1, \dots, b_K$ that are chosen so that the mean can be approximated arbitrarily well as a linear combination
$$\mu(t) = \sum_{k=1}^K \theta_k b_k(t), $$
as long as $K$ is large enough \parencite[see][Ch.\ 3]{RamsaySilverman2005}.
As we are dealing with complex valued functions, we use complex coefficients $\theta_k \in \mathbb{C}$ with real valued basis functions.

\textbf{[TODO: Motivate b-spline basis using Lisa's paper!]}


After choosing a basis representation,  we want to estimate complex coefficients $\theta_j \in \mathbb{C}$ so that the Full Procrustes mean of SRV curves is given by $\hat{\mu}(t) = \sum_{j=1}^k \hat{\theta}_j b_j(t) = b^T \hat{\theta}$:
\begin{align*}
    \hat{\mu} =& \argmax_{\theta : ||b^T\theta||=1} \sum_{i=1}^n \langle b^T\theta, q_i \rangle \langle q_i, b^T\theta \rangle \\
    =& \argmax_{\theta : ||b^T\theta||=1} \sum_{k,l} \sum_{i=1}^n \langle b_k \theta_k, q_i \rangle \langle q_i, b_l \theta_l \rangle \\
    =& \argmax_{\theta : ||b^T\theta||=1} \sum_{k,l} \bar{\theta}_k \theta_l \sum_{i=1}^n \langle b_k, q_i \rangle \langle q_i, b_l \rangle \\
    =& \argmax_{\theta : ||b^T\theta||=1} \theta^H S \theta \\
\end{align*}
where the matrix $S = \left\{ \sum_{i=1}^n \langle b_k, q_i \rangle \langle q_i, b_l \rangle \right\}_{k,l}$ has to be estimated from the observed SRV curves.
We can further simplify $S$ to
\begin{align*}
    S_{kl} =& \sum_{i=1}^n \int_0^1 \bar{b}_k(t) q_i(t) dt \int_0^1 \bar{q}_i(s) b_l(s) ds \\
    =& \int_0^1 \int_0^1 \bar{b}_k(t) \underbrace{\left( \sum_{i=1}^n q_i(t) \bar{q}_i(s) \right)}_{= n \, \hat{C}(s,t)} b_l(s) ds dt\\
    =& n \, \int_0^1 \int_0^1 \bar{b}_k(t) \hat{C}(s,t) b_l(s) ds dt\\
\end{align*}
with $\hat{C}(s,t) = \frac{1}{n} \sum_{i=1}^n q_i(s) \overline{q_i(t)}$ the sample analogue to the complex population covariance function $C(s,t) = \mathbb{E}[q(s)\overline{q(t)}]$.
We may estimate $C(s,t)$ via tensor product splines, so that $\hat{C}(s,t) = \sum_{k,l} \hat{\xi}_{kl} b_k(t) b_l(s)$, where $b_j(t)$, $j=1,\dots,k$ are the same real valued basis functions as used for the mean and $\hat{\xi}_{kl}$ are the estimated complex coefficients.
We can then further simplify $S_{kl}$
\begin{align*}
    S_{kl} =& n \, \int_0^1 \int_0^1 b_k(t) \left( \sum_{p,q} \hat{\xi}_{pq} b_q(t) b_p(s) \right) b_l(s) ds dt\\
    =& n \, \sum_{p,q} \hat{\xi}_{pq} \int_0^1 \int_0^1 b_k(t) b_q(t) b_p(s) b_l(s) ds dt\\
    =& n \, \sum_{p,q} \hat{\xi}_{pq} \langle b_k, b_q \rangle \langle b_p, b_l \rangle\\
    =& n \, \sum_{p,q} \hat{\xi}_{pq} g_{kq} g_{pl}
\end{align*}
where $g_{ij}$, $i,j = 1, \dots, k$ are the elements of the Gram matrix $G = bb^T$ with $G = \mathbb{I}_k$ in the special case of an orthogonal basis.
We can then write the write the matrix $S$ as a function of the estimated coefficient matrix $\hat{\Xi} = (\hat{\xi}_{ij})_{i,j = 1, \dots, k}$ :
\begin{align*}
    S =& n \, G \hat{\Xi} G
\end{align*}
The full Procrustes mean of SRV curves is then given by the solution to the optimization problem
\begin{align*}
    \hat{\mu} =& \argmax_{\theta} n \, \theta^H G \hat{\Xi} G \theta \quad \text{subj. to} \quad ||b^T \theta|| = 1 \\
    =& \argmax_{\theta : ||b^T\theta||=1} \, \theta^H G \hat{\Xi} G \theta \quad \text{subj. to} \quad \theta^H G \theta = 1
\end{align*}
One may solve this by using Lagrange optimization with the Langrangian
\begin{align*}
  \mathcal{L}(\theta,\lambda) =& \, \theta^H G \hat{\Xi} G \theta - \lambda ( \theta^H G \theta - 1)
\end{align*}



\section{Numerical Integration of the Procrustes Fits}
\label{sec:pfit}

\textbf{[TODO: Derivation as with Lisa.]}


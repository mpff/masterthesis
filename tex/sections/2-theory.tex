As a starting point, it is important to establish a notational and mathematical framework for the treatment of planar shapes.
While the restriction to the 2D case might seem a major one, it still covers all shape data extracted from e.g.\ imagery and is therefore very applicable in practice.
The outline of a 2D object may be naturally represented by a planar curves $\beta : [0,1] \rightarrow \mathbb{R}^2$ with $\beta(t) = (x(t),\, y(t))^T$, where $x(t)$ and $y(t)$ are the scalar-valued \textit{coordinate functions}.
Calculations in 2D, and in particular the derivation of the full Procrustes mean, are greatly simplified by using complex notation.
Going forward, we will therefore identify $\mathbb{R}^2$ with $\mathbb{C}$ and always use complex notation when representing a planar curve
$$\beta : [0,1] \rightarrow \mathbb{C}, \quad \beta(t) = x(t) + i\, y(t).$$
For reasons that will become apparent in Section \ref{theo:srv} we furthermore assume $\beta$ to be absolute continuous so that $\beta \in \mathcal{AC}([0,1],\, \mathbb{C})$.
\textbf{[Maybe better define spaces here? Hilbert space? Inner product? Basis expansin with complex coefficients. Or do that later on SRV level?]}




\section{Equivalence Classes and Shape Invariance}
\label{theo:inv}
The concept of shape is closely related to the concept of invariance under the transformations of scaling, translation and rotation.
When considering the shape of a curve, we additionally have to take into account invariance with respect to re-parametrisation.
This can be seen by noting that the curves $\beta(t)$ and $\beta(\gamma(t))$, with some re-parametrisation or \textit{warping function} $\gamma : [0,1] \rightarrow [0,1]$ monotonically increasing and differentiable, have the same image and therefore represent the same geometrical object.
We can say that the actions of translation, scaling, rotation and re-parametrisation are \textit{equivalence relations} with respect to shape, as each action leaves the shape of the curve untouched and only changes the way the shape is represented.
The shape of a curve can then be defined as the respective \textit{equivalence class}, i.e. the set of all possible shape preserving transformations of the curve.
As two equivalence classses are neccessarily either disjoint or identical, we can consider two curves as having the same shape, if they are elements of the same equivalence class \parencite[see][40]{SrivastavaKlassen2016}.

We now want to define an equivalence relation with respect to shape.
This is usually done using the notion of \textit{group actions} and \textit{product groups}, to which a brief introduction may be found in \cite[Chap.\ 3]{SrivastavaKlassen2016}.
Let us first consider how the discussed transformations act
individually on the set of parametrized planar curves with complex
representation $\beta : [0,1]
\rightarrow \mathbb{C}$:
\begin{itemize}
  \item
    The \textbf{translation} group $\mathbb{C}$ acts on $\beta$ by $(\xi, \beta) \xmapsto{\text{Trl}} \beta + \xi$, for any $\xi \in \mathbb{C}$.
    We can consider two curves as equivalent with respect to translation $\beta_1 \overset{\text{Trl}}{\backsim} \beta_2$, if there exists a complex scalar $\tilde\xi \in \mathbb{C}$ so that $\beta_1 = \beta_2  + \tilde\xi$.
    Then, for some function $\beta$, the related equivalence class with respect to translation is given by $[\beta]_{\text{Trl}} = \{\beta + \xi\, |\, \xi \in \mathbb{C}\}$.
  \item 
    The \textbf{scaling} group $\mathbb{R}^+$ acts on $\beta$ by $(\lambda, \beta) \xmapsto{\text{Scl}} \lambda \beta$, for any $\lambda \in \mathbb{R}^+$.
    We define $\beta_1 \overset{\text{Scl}}{\backsim} \beta_2$, if there exists a scalar $\tilde\lambda \in \mathbb{R}^+$ so that $\beta_1 = \tilde\lambda \beta_2$.
    An equivalence class is $[\beta]_{\text{Scl}} = \{\lambda\beta\,|\, \lambda \in \mathbb{R}^+\}$.
  \item 
    The \textbf{rotation} group $[0,2\pi]$ acts on $\beta$ by $(\theta, \beta) \xmapsto{\text{Rot}}  e^{i\theta} \beta$, for any $\theta \in [0,2\pi]$.
    We define $\beta_1 \overset{\text{Rot}}{\backsim} \beta_2$, if there exists a $\tilde\theta \in [0,2\pi]$ with $\beta_1 = e^{i\tilde\theta} \beta_2$.
    An equivalence class is $[\beta]_{\text{Rot}} = \{e^{i\theta}\beta\,|\, \theta \in [0,2\pi]\}$.
  \item 
    The \textbf{warping} group $\Gamma$ acts on $\beta$ by $(\gamma,\beta) \xmapsto{\text{Wrp}} \beta \circ \gamma$, for any $\gamma \in \Gamma$ with $\Gamma$ being the set of monotonically increasing and differentiable warping functions.
    We define $\beta_1 \overset{\text{Wrp}}{\backsim} \beta_2$, if there exists a warping function $\tilde\gamma \in \Gamma$ with $\beta_1 = \beta_2 \circ \tilde\gamma$.
    An equivalence class is $[\beta]_{\text{Wrp}} = \{\beta \circ \gamma\,|\, \gamma \in \Gamma\}$.
\end{itemize}
In a next step, we can consider how these transformations act in concert and whether they \textit{commute}, that is, whether the order of applying the transformations changes outcomes.
Consider for example the actions of the rotation and scaling product group $\mathbb{R}^+ \times [0,2\pi]$ given by $(\lambda, (\theta, \beta)) \xmapsto{\text{Scl} + \text{Rot}} \lambda e^{i\theta} \beta$.
These clearly commute, because the order of applying rotation or scaling do not make a difference, as $\lambda(e^{i\theta}\beta) = e^{i\theta}(\lambda\beta)$.
However, the joint actions of scaling and translation do not commute, as $\lambda(\beta + \xi) \neq \lambda\beta + \xi$, with the same holding for rotation and translation.
It follows that the order of translating and rotating or scaling matters and when defining such a joint action, one usually takes the translation to act on the already scaled and rotated curve.
\textbf{[Maybe better define rot/scaling on the centered and retranslated curve, as in \cite{Stoecker2021}?]}
\begin{definition}[Euclidean similarity transformation] 
  We define an \textbf{Euclidean similarity transformation} of a curve $\beta :
  [0,1] \rightarrow \mathbb{C}$ as the joint action of scaling, rotation, and
  translation by 
  $$((\xi, \lambda, \theta), \beta) \mapsto \lambda e^{i\theta} \beta + \xi,$$
  with $\xi \in \mathbb{C}$, $\lambda \in \mathbb{R}^+$, and $\theta \in
  [0,2\pi]$ \parencite[see][62]{DrydenMardia2016}.
\end{definition}
\noindent With respect to the action of re-parametrization, we can note that it
necessarily commutes with all Euclidean similarity transformations, as those
only act on the image of $\beta$, while the former only acts on the
parametrization.
Putting everything together, we can finally give a formal definition of the
shape of a planar curve.
\begin{definition}[Shape]
  The \textbf{shape} of a planar curve $\beta : [0,1] \rightarrow \mathbb{C}$
  is given by its equivalence class with respect to all Euclidean similarity
  transformations and re-parametrizations
  $$ [\beta] = \left\{\lambda e^{i\theta}(\beta \circ \gamma) + \xi\,|\, \xi
  \in \mathbb{C},\, \lambda \in \mathbb{R}^+,\, \theta \in [0,2\pi],\, \gamma
  \in \Gamma\right\}. $$
  The \textbf{shape space} is then given by the corresponding quotient space 
  $$\mathcal{AC}([0,1], \mathbb{C}) \big/ \mathbb{C} \rtimes \left( \mathbb{R}^+ \times
  [0,2\pi] \right) \times \Gamma = \left\{[\beta]\,|\,\beta \in
  \mathbb{C}^{[0,1]}\right\},$$
  where the symbol \enquote{$\rtimes$} denotes a semi-direct product, i.e. that
  the translation group acts \enquote{after} scaling and rotation and where we assumed the $\beta$'s to be absolutely continuous \textbf{[This notation is not very clear.]}
  \parencite[for details see][Chapter 3]{SrivastavaKlassen2016}.
\end{definition}

\noindent We now want to construct a distance function in shape space.
As equivalence classes as the elements of shape spaces are quite complex objects, this is usually done by calculating distances in the underlying functional space, where one optimizes over all possible elements of both equivalence classes.
For example, when assuming $\beta_1$, $\beta_2 \in \mathbb{L}^2([0,1],\, \mathbb{C})$ we might calculate distances in shape space by optimizing over their $\mathbb{L}^2$-distance
$$d([\beta_1], [\beta_2]) = \inf_{\tilde \beta_1 \in [\beta_1],\,\tilde \beta_2 \in [\beta_2]} d_{\mathbb{L}^2}(\tilde \beta_1, \tilde \beta_2) = \inf_{\tilde\beta_1 \in [\beta_1],\, \tilde\beta_2 \in [\beta_2]} || \tilde\beta_1 - \tilde\beta_2 ||.$$
However, this approach runs into problems, when considering whether all shape-preserving transformations act by isometries on this distance, i.e.\ whether equal changes in translation, rotation, scaling and re-parametrization to both curves change their distance. 
As it turns out, neither re-parametrization nor scaling are distance preserving when using the $\mathbb{L}^2$-distance, as $||\beta_1 \circ \gamma - \beta_2 \circ \gamma|| \neq ||\beta_1 - \beta_2||$ and $||\lambda \beta_1 - \lambda \beta_2|| \neq ||\beta_1 - \beta_2||$.



\section{The SRV Framework and the Elastic Distance}
\label{theo:srv}
Let us first consider only the issues of re-parametrization and translation.
A solution proposed by \cite{SrivastavaEtAl2011} is the use of the \textit{square-root-velocity} (SRV) framework and an \textit{elastic} metric, the Fisher-Rao Riemannian metric, which is isometric with respect to re-parametrization of the input curves.
The calculation of the Fisher-Rao metric between two curves can be simplified by using the usual $\mathbb{L}^2$-metric between their respective SRV curves.

\begin{definition}[Elastic distance \parencite{SrivastavaEtAl2011}]
  For $\beta_1$, $\beta_2 \in \mathcal{AC}([0,1],\,\mathbb{C})$ with $[\beta_1]_{\text{Trl+Wrp}}$, $[\beta_2]_{\text{Trl+Wrp}}$ their respective equivalence classes modulo translation and warping, the \textbf{elastic distance} between $[\beta_1]_{\text{Trl+Wrp}}$ and $[\beta_2]_{\text{Trl+Wrp}}$ is given by
  $$d([\beta_1]_{\text{Trl+Wrp}},\, [\beta_2]_{\text{Trl+Wrp}}) = \inf_{\gamma \in \Gamma} || q_1 - (q_2 \circ \gamma) \sqrt{\dot\gamma} ||,$$
  where $q_1, q_2$ denote the corresponding \textbf{square-root-velocity functions} (SRVFs) given by
$$ \quad q(t) = \frac{\dot{\beta}(t)}{\sqrt{|| \dot{\beta}(t) ||}} \quad \text{for} \,\, \dot{\beta}(t) \neq 0, \,\, \text{with} \,\, q \in \mathbb{L}^2([0,1], \mathbb{C}). $$
\end{definition}

\noindent When working in the SRV framework, the original curve $\beta$ can always be obtained from its SRVF up to translation, by the back transformation $\beta(t) = \beta(0) + \int_0^t q(s) || q(s) || ds$.
As this representation makes use of derivatives, any curve $\beta$ that has a SRVF must fulfill some kind of differentiability constraint, where in this case this holds for any absolutely continuous $\beta \in \mathcal{AC}([0,1],\, \mathbb{C})$ \parencite[see][134]{SrivastavaKlassen2016}.
The SRVFs are considered elements of a Hilbert space, given by $\mathcal{L}^2([0,1],\,\mathbb{C})$ equipped with the complex inner product $\langle \cdot, \cdot \rangle$ and corresponding norm $||\cdot||$, where the complex inner product of $q,q' \in \mathcal{L}^2([0,1],\,\mathbb{C})$ is given by
$$ \langle q, q' \rangle = \int_0^1 \overline{q(t)} q'(t) dt \,, $$
with $\overline{z} = \operatorname{Re}(z) - i \operatorname{Im}(z)$ denoting the complex conjugate.

As we can always recover the original curve up to translation, the SRV representation holds all relevant information about the shape of a curve.
This means we can perform any shape analysis equivalently on SRV level, as we can later always transform the results of e.g.\ a mean calculation back to original curve level.
\textbf{[Show this?]}

\begin{lemma}
  The actions of the translation, scaling, rotation, and re-parametrization groups commute on SRV level.
\end{lemma}

\begin{proof} The SRVF $\tilde q(t)$ of  $\tilde\beta(t) = \lambda e^{i\theta}\beta\left(\gamma(t)\right) + \xi$ is given by
$$ \tilde q (t) 
  = \frac{\dot{\tilde\beta}(t)}{\sqrt{| \dot{\tilde\beta}(t) |}} 
  = \frac{\lambda e^{i\theta} \dot\beta\left(\gamma(t)\right) \dot\gamma(t)}{\sqrt{||\lambda e^{i\theta} \dot\beta\left(\gamma(t)\right) \dot\gamma(t)||}} 
  = \sqrt{\lambda} e^{i\theta} \frac{\dot\beta\left(\gamma(t)\right)}{\sqrt{||\dot\beta\left(\gamma(t)\right)||}} \sqrt{\dot\gamma(t)} 
  = \sqrt\lambda e^{i\theta} \left( q \circ \gamma \right) \sqrt{\dot\gamma(t)},$$
where the result is the same, irrespective of the order of applying the transformations.
\end{proof}

\noindent It follows, that the individual transformations translate to SRV level by i.) $(\xi, q) \xmapsto{\text{Trl}} q$, ii.) $(\lambda, q) \xmapsto{\text{Scl}} \sqrt\lambda q$, iii.) $(\theta, q) \xmapsto{\text{Rot}} e^{i\theta} q$, iv.) $(\gamma, q) \xmapsto{\text{Wrp}} (q \circ \gamma) \sqrt{\dot\gamma}$, where, in particular, the SRVF is invariant to translations of the original curve.
The invariance with respect to translation is a nice property of the SRV framework, as it means that we can identify the shape of a curve as the equivalence class of its respective SRVF modulo scaling, rotation, and warping, but we do not need to consider translation.
\begin{definition}[Shape (SRV level)]
  The \textbf{shape} of an absolutely continuous, planar curve $\beta \in \mathcal{AC}([0,1],\, \mathbb{C})$ is given by the equivalence class of its SRV representation $q = \frac{\dot\beta(t)}{\sqrt{|\dot\beta(t)|}} \in \mathbb{L}^2([0,1],\,\mathbb{C})$ modulo scaling, rotation and re-parametrization
  $$[q] = \left\{\sqrt\lambda e^{i\theta}(q \circ \gamma) \sqrt{\dot\gamma(t)} \,|\, \lambda \in \mathbb{R}^+,\, \theta \in [0,2\pi],\, \gamma
  \in \Gamma\right\},$$
  and it holds that $[\beta] \cong [q]$. The \textbf{shape space} can then be identified with 
  $$\mathbb{L}^2 \big/ G = \mathbb{L}^2([0,1], \mathbb{C}) \big/ \mathbb{R}^+ \times [0,2\pi] \times \Gamma = \left\{[q]\,|\,q \in \mathbb{L}^2({[0,1],\,\mathbb{C})}\right\}.$$
  \textbf{[Note: Definition should have the closure of [q]!]}
\end{definition}



\section{The Elastic Full Procrustes Distance for Planar Curves}
\label{theo:proc}
The \textit{full Procrustes distance} is a widely used distance function in
classical shape analysis that allows for scale and rotation invariant shape
distance calculation.
\begin{definition}[Full Procrustes distance]
  The \textbf{full Procrustes distance} between the shapes $[\beta_1]$,
  $[\beta_2]$ of two continously differentiable $\beta_1$, $\beta_2$ with
  $\beta_i : [0,1] \rightarrow \mathbb{C}$ is given by 
    $$d_F([\beta_1], [\beta_2]) = \inf_{\lambda \in \mathbb{R}^+,\, \theta \in
    [0,2\pi]} ||\tilde{\beta}_1 - \lambda e^{i\theta} \tilde{\beta}_2||, $$
    where $\tilde\beta_{1,2}$ are the centered, unit-length pre-shapes of
    $\beta_{1,2}$.
\end{definition}
\begin{definition}[Full Procrustes mean]
    The \textbf{full Procrustes mean} shape for a sample of landmark
    configurations $X_i$ ($i = 1,\dots,n$) is then given by the equivalence
    class [$\hat\mu_F$] of a landmark configuration that minimizes the sum of
    squared full Procrustes distances
    $$\hat{\mu}_F = \arginf_{\mu} \sum_{i=1}^n d_F([\mu], [X_i])^2, $$
    where $\mu$ is assumed centered and normalized
    \parencites[see][71,114]{DrydenMardia2016}.
\end{definition}

\textbf{[Define an intersection of the quotient space by projections. Proof that this is good. Procrustes Fits.]}

Moreover, if the original curve is of unit length the SRV curve will be automatically normalized \textbf{[Keep this for later]}:
$$ ||q|| = \sqrt{\langle q, q \rangle} = \sqrt{ \int_0^1 \overline{q(t)} q(t) \, dt } = \sqrt{ \int_0^1 |q(t)|^2 \, dt} = \sqrt{\int_0^1 |\dot{\beta}(t)| \, dt} = \sqrt{1} = 1. $$


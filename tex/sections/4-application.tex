\label{sec:4}
In this chapter the proposed methods will be applied and verified.
\cref{sec:4-means} offers a comparison between the elastic full Procrustes mean, the elastic mean and the full Procrustes mean over two datasets.
\cref{sec:4-penalty} briefly discusses the effect of the penalty parameter on the estimation.
\cref{sec:4-pfits} mentions pitfalls when plotting the mean together with the aligned curves and discusses outliers.
Finally, \cref{sec:4-tounges} applies the proposed methods to an empirical dataset of tongue contours, which was kindly provided by Prof.\ Dr.\ Marianne \textsc{Pouplier}.


\section{Comparison to the Elastic and the Full Procrustes Mean}
\label{sec:4-means}
In this section we will estimate and compare the elastic mean, the full Procrustes mean and the elastic full Procrustes mean for two datasets.
The first dataset is the \texttt{digits3.dat} dataset provided in the \texttt{shapes} package \parencite{shapes} and originally collected by \cite{Anderson1997}, which was already used \cref{sec:2,sec:3} for illustrative purposes.
It consist of 30 handwritten digits \enquote*{3}, each of which was sampled in a regular fashion at 13 points along the digit, leading to sparse but somewhat regular observations.
The second dataset consists of ten simulated spirals, each of which is a sample of the curve $\beta(t) = t \cos(13t) + \iu \cdot t \sin(13t)$ evaluated $n_i \in [10,15]$ times over a noisy grid, with additional noise applied to the output, leading to sparse and irregular observations.
The code simulating these spirals was adapted from the \texttt{compute\_elastic\_mean} function's documentation in the \texttt{elasdics} package \parencite{elasdics}.
Because the spirals \enquote{speed up} w.r.t.\ $t$ towards the end ($t = 1$), they start out quite densely sampled but become increasingly sparser, making the mean estimation close to $t=1$ a challenge.
The datasets are considered in two settings:
In the first, each dataset is considered as is, meaning that all curves are centred and similarly aligned.
In the second, for each curve $\beta_i$ a random Euclidean similarity transform with translation $\xi_i \sim \mathcal{U}([\xi_\mathrm{min}, \xi_\mathrm{max}])$, rotation $\theta_i \sim \mathcal{U}([0,2\pi))$ and scaling $\lambda_i \sim \mathcal{U}([0.5,1.5])$ is drawn, where $\xi_\mathrm{min}, \xi_\mathrm{max}$ are set respectively to $\pm 60$ and $\pm 2$ for the digits and spirals.
The curves are then transformed by $\lambda_i e^{\iu \theta_i} \beta_i + \xi_i$.

The four sets of data curves and means, are shown in \cref{fig:4-means}.
\begin{figure}
  \centering
  \begin{subfigure}{\textwidth}
    \centering
    \inputTikz{41-digit3-means}
    \caption{Means for \texttt{digits3.dat}.}
    \label{fig:41-digit3-means-a}
  \end{subfigure}\vspace{0.66em}\\
  \begin{subfigure}{\textwidth}
    \centering
    \inputTikz{41-digit3-means-rot}
    \caption{Means for \texttt{digits3.dat} with random Euclidean similarity transform applied.}
    \label{fig:41-digit3-means-b}
  \end{subfigure}\vspace{0.66em}\\
  \begin{subfigure}{\textwidth}
    \centering
    \inputTikz{41-spiral-means}
    \caption{Means for simulated sparse spirals.}
    \label{fig:41-spiral-means-a}
  \end{subfigure}\vspace{0.66em}\\
  \begin{subfigure}{\textwidth}
    \centering
    \inputTikz{41-spiral-means-rot}
    \caption{Means for simulated sparse spirals with random Euclidean similarity transform applied.}
    \label{fig:41-spiral-means-b}
  \end{subfigure}
  \caption{Comparison of three mean types: Elastic mean (blue), full Procrustes mean (green) and elastic full Procrustes mean (red), estimated over four sets of data curves (grey).
  Each mean is estimated as polygonal (light, 16 knots) and smooth (dark, 13 knots), where in the estimation of the two Procrustes means a 2nd order penalty was applied.
  Data: See \cref{sec:4-means}}
  \label{fig:4-means}
\end{figure}
Here, the elastic mean (blue) is estimated using \texttt{compute\_elastic\_mean} from the \texttt{elasdics} package.
The full Procrustes (green) and the elastic full Procrustes mean (red) are estimated using the methods proposed in \cref{sec:2,sec:3}.
For the full Procrustes mean the estimation is stopped before the warping alignment step during the first iteration.
Note that the full Procrustes mean calculated in this way is not exactly a minimizer of the sum of squared full Procrustes distance defined in \cref{def:2-fpdist}, but instead is a minimizer of the sum of squared full Procrustes distances on SRV level.
When comparing the different mean types in \cref{fig:4-means} we can see that, unlike the elastic mean, both Procrustes means are invariant with respect to all Euclidean similarity transforms, as the estimated mean is the same for the transformed and original datasets.
However for this very same reason, both Procrustes means hold no information about the scale or rotation of the original curves, as they are of unit-length and have a rotation dependent on the eigendecomposition of the covariance surface.
The elastic mean is invariant only with respect to re-parameterisation and translation, so that its scale and rotation match the original data curves.
It can therefore be meaningfully plotted together with the original curves, when they are centred.

\cref{fig:4-means} provides two important validation checks for the estimation procedure proposed in \cref{sec:2,sec:3}.
Firstly, the estimated elastic full Procrustes mean is invariant to all Euclidean similarity transforms, as the mean shapes do not change with transformations of the input curves.
Secondly, the estimated elastic full Procrustes mean shapes are very comparable to estimated elastic mean shape, when considering the untransformed curves.
This is especially notable when comparing the means in \cref{fig:4-means} (a) where the prominent \enquote{notch} in the centre of the mean shape is similarly pronounced for the elastic and the elastic full Procrustes means, but not for the (non-elastic) full Procrustes mean.
Taken together, this shows that the proposed mean estimation method provides elastic mean estimates in the setting of sparse and irregular curves, which are invariant to all shape-preserving transformations.


\section{Effect of the Penalty Parameter on Estimated Means}
\label{sec:4-penalty}
The elastic full Procrustes mean is given by the leading eigenfunction of the complex covariance surface, which was estimated using tensor product P-splines.
As a consequence, the order of the roughness penalty applied in the estimation of the covariance surface directly influences the shape of the estimated mean function.
This can be seen in \cref{fig:4-penalty}, where a smooth mean was estimated for different penalties.
\begin{figure}
  \centering
  \begin{subfigure}{\textwidth}
  \inputTikz{41-digit3-pen}
  \end{subfigure}
  \begin{subfigure}{\textwidth}
  \inputTikz{41-spiral-pen}
  \end{subfigure}
  \caption{Elastic full Procrustes mean under different penalties.
  Estimated using no penalty (left) and order 0/1/2 penalties (centre-left/centre-right/right) respectively, as well as 13 equidistant knots and linear B-splines on SRV level.
  Data: See \cref{fig:41-digit3-means-b,fig:41-spiral-means-b}}
  \label{fig:4-penalty}
\end{figure}
Here, in particular, the spiral mean shapes help to illustrate the penalty effect.
As mentioned in \cref{sec:4-means}, the spirals are evaluated in a way that makes them very sparse towards the end, leading to unstable mean estimates in that region.
By penalising $p$-th order differences between neighbouring coefficients, the penalty helps to stabilise the covariance estimation (and thereby the mean) in regions where observations are sparse and where the shape of the function is consequently dominated by the penalty.

The mean function on the far left in \cref{fig:4-penalty} was estimated using no penalty.
We can see that the spiral shape is estimated well in the beginning (central part), where observations are dense, but becomes increasingly unstable and wriggly towards the end (outer part).
The mean function on the centre-left was estimated using a zero order penalty.
A zero order penalty on a B-spline basis can be interpreted as a ridge-penalty on the basis coefficients, i.e.\ the basis coefficients (and therefore the estimated function) get shrunk towards zero in areas where observations are sparse.
It is important to note that the penalised covariance estimation is performed on SRV level and not on data curve level, which means the ridge penalty shrinks the estimated SRV mean towards zero.
However, this does not imply that the estimated mean on data curve level also gets shrunk towards zero.
In fact, a ridge penalty on SRV level only shrinks the \emph{absolute velocity} of the mean on data curve level towards zero, but should not directly influence the \emph{direction} of the mean curve.
\footnote{While the penalty acts on the estimated covariance surface, its effect seems to translate more or less directly to its leading eigenfunction.}
This can be seen, when comparing the zero order penalty spiral mean (centre-left) to the higher order penalty spiral means (right).
As a consequence of the decreased absolute velocity the former is relatively \enquote{shorter} towards the end, when compared to the beginning of the spiral.
The higher order penalties may be interpreted as smoothing the SRV mean function towards a polynomial of degree $p-1$ in areas where the penalty dominates, which means a constant function for the first order penalty and a linear function for the second order penalty \parencite[see e.g.][435]{FahrmeierEtAl2013}.
Looking at the means for $p = 1,2$ in \cref{fig:4-penalty}, these differences are already hard to spot.
The order two mean (right) is slightly longer towards the end, indicating a speed up, which is caused by the penalisation towards a global linearly (increasing) velocity, compared to a more conservative penalisation towards a global constant velocity. 


\section{Elastic Full Procrustes Fits and Outliers}
\label{sec:4-pfits}
Although the elastic full Procrustes mean does not share the rotation, scale and translation of the input curves, it is still possible to visually compare it to them, by plotting the mean together with the elastic full Procrustes fits.
In this thesis, the elastic full Procrustes fits are calculated on SRV level as $\widetilde q^\mathrm{EP} = (\omega^\mathrm{opt}\cdot \widetilde q \circ \gamma^\mathrm{opt} ) \sqrt{\dot\gamma^\mathrm{opt}}$, where the optimal rotation and scaling alignment $\omega^\mathrm{opt} = \lambda^\mathrm{opt} e^{\iu \theta^\mathrm{opt}}$ and optimal warping alignment $\gamma^\mathrm{opt}$ of each curve to the mean are taken from the last iteration of the mean estimation step.
In \cref{fig:4-pfits-curve} the elastic full Procrustes mean and fits are plotted for the digits \enquote*{3} and simulated spiral datasets discussed in \cref{sec:4-means}.
\begin{figure}
  \centering
  \begin{subfigure}{\textwidth}
    \begin{subfigure}{0.48\textwidth}
      \inputTikz{4-digits3-pfits}
    \end{subfigure}\hfill%
    \begin{subfigure}{0.48\textwidth}
      \inputTikz{4-spirals-pfits}
    \end{subfigure}
    \caption{Centred elastic full Procrustes fits (grey) and smooth means (red).}
    \label{fig:4-pfits-curve}
  \end{subfigure}
  \begin{subfigure}{\textwidth}
    \begin{subfigure}{0.48\textwidth}
      \inputTikz{4-digits3-pfits-srv}
    \end{subfigure}\hfill%
    \begin{subfigure}{0.48\textwidth}
      \inputTikz{4-spirals-pfits-srv}
    \end{subfigure}
    \caption{Elastic full Procrustes fits (grey) and piece-wise-linear means (red) on SRV level.}
    \label{fig:4-pfits-srv}
  \end{subfigure}
  \caption{Elastic full Procrustes means and fits for digits (left) and spirals (right). Parameters: Second order penalty and linear B-splines, with 13 (left) and 19 (right) equidistant knots. Data: See \cref{fig:41-digit3-means-b,fig:41-spiral-means-b}}
  \label{fig:4-pfits}
\end{figure}
This alignment works very well for the simulated spirals and is more fuzzy for the digits, due to their greater variability in shape. 

In general one has to be careful when interpreting these plots.
For example, it can be seen that the estimated mean does not necessarily follow the \enquote{centre of mass} of the aligned curves. 
One reason for this is that the mean is calculated on SRV level, meaning we may set an arbitrary translation for the mean on data curve level.
Here, the Procrustes fits and the mean function are centred, but a natural alternative might be to have all curves start at the origin.
Both choices lead to very different but equally valid visual representations.
Likewise, we can see that the mean on SRV level is also not a simple functional mean of the aligned curves.
This may be caused by the scaling alignment, which tends to favour sightly shrinking curves to decrease their distance to the mean, as shown by values of $\lambda^\mathrm{opt} < 1$ in \cref{fig:4-bopts}.
This shrinkage is more noticeable in the \texttt{digits3.dat} for which four outlier shapes are shown in \cref{fig:4-outliers}, which were classified using their distance to the mean. 
\begin{figure}
  \centering
  \begin{subfigure}[t]{0.59\textwidth}
    \begin{subfigure}[t]{0.40\textwidth}
      \centering
      \inputTikz{4-outlier-dist}
    \end{subfigure}
    \begin{subfigure}[t]{0.60\textwidth}
      \centering
      \inputTikz{4-outlier}
    \end{subfigure}
    \caption{Distribution of elastic full Procrustes distances to the estimated mean (red) with four largest outliers (grey).}
  \label{fig:4-outliers}
  \end{subfigure}
  \begin{subfigure}[t]{0.39\textwidth}
    \centering
    \inputTikz{4-outlier-bopt}
    \caption{Distribution of the optimal scaling alignment parameter of each elastic Procrustes fit.}
    \label{fig:4-bopts}
  \end{subfigure}
  \caption{Elastic full Procrustes distances, outliers and estimated optimal scaling. Parameters: See \cref{fig:4-pfits}. 
  Data: See \cref{fig:41-digit3-means-b,fig:41-spiral-means-b}}
\end{figure}
Robustness to outliers is a prominent feature of the full Procrustes mean, when compared to the normal and partial Procrustes means.
This comes from the discussed shrinkage, seen for example for outliers $1$ and $4$ in \cref{fig:4-outliers}.
When considering how this shrinkage might influence the warping alignment over the elastic full Procrustes fits, we can note that a bad scaling alignment is not immediately problematic as the warping alignment only takes into account the \emph{relative} distances over $t$ between the aligned and the mean curve
While a bad rotation alignment will most likely always lead to a bad warping alignment, these outliers are only problematic when considering the alignment of single curves, as their influence on the estimated overall mean shape will always be limited by a shrinkage to zero.


\section{Analysis of Variability in Tongue Shapes}
\label{sec:4-tounges}
Mean and distance estimation for planar shapes extracted from imagery is a typical use-case for the proposed methods.
In this section, tongue contours obtained from ultrasound recordings are analysed in their phonetic context.
The data was kindly provided by Prof.\ Dr.\ Marianne \textsc{Pouplier} and was gathered in an experimental setting from 6 native German speakers, each of whom was recorded speaking the same set of ficticious words over multiple repetitions.
Each word is a combination of two flanking vowels such as 'aa' or 'ii' and one consonant such as 'd' or 'l', leading to words such as 'pada', 'pidi', 'pala' or 'pili'.
The tongue contours follow the centre of the tongue and are extracted from the ultrasound recordings at times corresponding to the centre of consonant articulation, which is estimated from the acoustic signal.
Furthermore, care was taken to observe the tongues over their full length, from their tips to the Hoyid bone (German: Zungenbein),  which means we can naturally treat them as planar curves $\beta : [0,1] \rightarrow \mathbb{C}$ \parencite[see][]{consulting}.
While there is earlier work on functional data approaches to tongue shape analysis \parencite{CederbaumEtAl2016,Davidson2006,PouplierEtAl2014}, these approaches do not account for shape invariance in their estimation.
Procrustes analysis can account for anatomical differences between speakers, such as their differing sizes, or measurement inconsistencies between iterations, such as differences in orientation of the ultrasound device.
At the same time, the tongue is a flexible muscle capable of stretching, bending and compressing itself, providing some indication that an elastic analysis is appropriate. 
Therefore, the tongue shapes will be analysed using the elastic full Procrustes mean and distance.

The dataset we analyse consists of tongue shapes during articulation of the consonants 'd', 'l', 'n' and 's'.
For each consonant we consider two flanking vowels, 'aa' and 'ii', resulting in eight unique combinations of vowels and consonants.
There are multiple sources of variation in the dataset: i.) variation over the consonants, ii.) variation over the vowel context, iii.) variation over the speakers and iv.) variation over multiple repetitions of the same word by the same speaker.
We want to analyse which of these factors are relatively more important for tongue shape than others.
Does tongue shape vary more strongly between the vowel contexts than between different consonants?
Is the tongue shape of an individual speaker consistent over multiple repetitions of the same word?
How strongly does each speakers individual way of speaking influence the tongue shape?

We want to answer these questions using only mean and distance calculations.
For this, the dataset is grouped hierarchically, as displayed in \cref{fig:4-tongue-means}.
In a first step, we can group the dataset by vowels and by consonants and estimate means over all observed tongue curves belonging to a specific consonant or vowel (right column and bottom row).
In a second step, we can group the dataset by unique combinations of vowel and consonant (central eight plots).
In a third step, we can estimate six per-speaker means (coloured) for each group in the first and second steps.
Then, to get a measure for the variability of tongue shapes inside a specific group, each curves elastic full Procrustes distance to its respective group mean is calculated and their distribution is compared (see \cref{fig:4-tongue-dists}).
\begin{figure}
  \centering
  \advance\leftskip-1cm
  \begin{subfigure}[t]{0.72\textwidth}
    \centering
    \inputTikz{44-tounge-means}
    \centering
    \caption{Total (black, dashed) and per-speaker means (coloured) by vowels and consonants, with per-speaker elastic full Procrustes fits (grey). 
    Per-speaker means and -fits are rotationally aligned to the respective total means.}
    \label{fig:4-tongue-means}
  \end{subfigure}%
  \begin{subfigure}[t]{0.40\textwidth}
    \centering
    \inputTikz{44-tounge-dists}
    \centering
    \caption{Distribution of elastic full Procrustes distances to respective mean (see \cref{fig:4-tongue-means}) total (black) and by speaker (coloured), where \enquote*{t} indicates distances to the total mean.}
    \label{fig:4-tongue-dists}
  \end{subfigure}
  \caption{Right column and bottom row correspond to means and distances over all vowels or all consonants respectively, with the bottom right means corresponding to global and global per-speaker means and distances.
    Parameters: Estimated using 13 equidistant knots, linear B-splines on SRV level and a 2nd order penalty.
    Data: Prof.\ Dr.\ M.\ \textsc{Pouplier}.}
  \label{fig:4-tounges}
\end{figure}

We can note that grouping by vowel (comparing the right to the other columns) consistently decreases variability more so than grouping by consonant (comparing the bottom to the other rows). 
This indicates that vowel context has a greater influence on tongue shape, when compared to the actual  consonant that was spoken at time of measurement, which might already be surprising for someone not familiar with the subject.
This is also confirmed visually, when considering the stark difference in estimated mean shapes between vowels and comparing it to the difference in mean shapes between consonants.
In fact, when considering the distance distributions for total means (black), we can note the mean distance does hardly changes when grouping by consonant (comparing the bottom to the other rows).
This is in contrast to the per-speaker means (coloured), where grouping by consonant seems to decrease the variability at least slightly.
It seems that without accounting for the individual speaker, the consonant only explains very little in terms of variability of the tongue shapes.
This indicates that there is no strong common effect of the spoken consonant on tongue shape, but rather that this effect depends on each speaker individually.
Finally, when considering the grouping by single words (central eight plots), we can note that the variability measured over the per-speaker means is consistently smaller than the variability of the total means, indicating that tongue shape is very consistent on a per-speaker basis.


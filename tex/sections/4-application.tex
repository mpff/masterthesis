\label{sec:4}
In this chapter the proposed methods will be applied and verified.
\cref{sec:4-means} offers a comparison between the elastic full Procrustes mean, the elastic mean and the full Procrustes mean over two datasets, with the goal of illustrating some properties of the elastic full Procrustes mean.
\cref{sec:4-penalty} will briefly discuss the effect of the penalty parameter on the estimation.
\cref{sec:4-pfits} discusses \todo{Anpassen}\dots .
Finally, \cref{sec:4-tounges} applies the proposed methods to an empirical dataset of tounge shapes, which was kindly provided by \todo{Phonetics Data Citations}\dots.


\section{Comparison to the Elastic and the Full Procrustes Mean}
\label{sec:4-means}
In this section we will estimate and compare the elastic mean, the full Procrustes mean and the elastic full Procrustes mean for two datasets.
The first dataset is the \texttt{digits3.dat} dataset provided in the \texttt{shapes} package \parencite{shapes} and originally collected by \cite{Anderson1997}, which has already been used throughout this thesis for illustrative purposes.
It consist of 30 handwritten digits \enquote*{3}, each of which was sampled in a regular fashion at 13 points along the digit, leading to sparse but somewhat regular observations.
The second dataset consists of ten simulated spirals, each of which is a sample of the curve $\beta(t) = t \cos(13t) + \iu \cdot t \sin(13t)$, evaluated $m_i \in [10,15]$ times over a noisy grid, with additional noise applied to the output, leading to sparse and irregular observations.
The code simulating these spirals was adapted from the \texttt{compute\_elastic\_mean} function's documenation in the \texttt{elasdics} package \parencite{elasdics}.
Because the spirals \enquote{speed up} w.r.t.\ $t$ towards the end ($t = 1$), they start out quite densly sampled but become increasingly sparser, making the mean estimation, especially close to $t=1$, a challenge.
The datasets are considered in two settings:
In the first, each dataset is considered as is, which means that all curves are centered and similarly aligned.
In the second, for each curve $\beta_i$ a random Euclidean similarity transform with translation $\xi_i \sim \mathcal{U}([\xi_\mathrm{min}, \xi_\mathrm{max}])$, rotation $\theta_i \sim \mathcal{U}([0,2\pi))$ and scaling $\lambda_i \sim \mathcal{U}([0.5,1.5])$ is drawn, where $\xi_\mathrm{min}, \xi_\mathrm{max}$ are set respectively to $\pm 60$ and $\pm 2$ for the digits and spirals.
The curves are then transformed by $\lambda_i e^{\iu \theta_i} \beta_i + \xi_i$.

The four sets of data curves and means, are show in \cref{fig:4-means}.
\begin{figure}
  \centering
  \begin{subfigure}{\textwidth}
    \centering
    \inputTikz{41-digit3-means}
    \caption{Means for \texttt{digits3.dat}.}
    \label{fig:41-digit3-means-a}
  \end{subfigure}\vspace{0.66em}\\
  \begin{subfigure}{\textwidth}
    \centering
    \inputTikz{41-digit3-means-rot}
    \caption{Means for \texttt{digits3.dat} with random Euclidean similarity transform applied.}
    \label{fig:41-digit3-means-b}
  \end{subfigure}\vspace{0.66em}\\
  \begin{subfigure}{\textwidth}
    \centering
    \inputTikz{41-spiral-means}
    \caption{Means for simulated sparse spirals.}
    \label{fig:41-spiral-means-a}
  \end{subfigure}\vspace{0.66em}\\
  \begin{subfigure}{\textwidth}
    \centering
    \inputTikz{41-spiral-means-rot}
    \caption{Means for simulated sparse spirals with random Euclidean similarity transform applied.}
    \label{fig:41-spiral-means-b}
  \end{subfigure}
  \caption{Comparison of three mean types: Elastic mean (blue), full Procrustes mean (green) and elastic full Procrustes mean (red), estimated over four sets of data curves (grey).
  Each mean is estimated as polygonal (light, 16 knots) and smooth (dark, 13 knots), where in the estimation of the two Procrustes means a 2nd order penalty was applied.
  Data: See \cref{sec:4-means}}
  \label{fig:4-means}
\end{figure}
Here, the elastic mean (blue) is estimated using \texttt{compute\_elastic\_mean} from the \texttt{elasdics} package.
The full Procrustes (green) and the elastic full Procrustes mean (red) are estimated using the methods proposed in \cref{sec:2,sec:3}, where for the full Procrustes mean the estimation is stopped before the warping alignment step during the first iteration.
Note that the full Procrustes mean calculated in this way is not exactly\todo{Vielleicht sogar doch "exactly?"} a minimizer of the sum of squared full Procrustes distance defined in \cref{def:2-fpdist}, but instead is a minimizer of the sum of squared full Procrustes distances on SRV level.
When comparing the different mean types in \cref{fig:4-means} we can see that, unlike the elastic mean, both Procrustes means are invariant with respect to all Euclidean similarity transforms, as the estimated mean is the same for the transformed and original datasets.
However for this very same reason, both Procrustes means hold no information about the scale or rotation of the original curves, as they are of unit-length and have a rotation dependent on the eigendecomposition of the covariance surface.
The elastic mean is invariant only with respect to re-parametrization and translation, so that its scale and rotation match the original data curves.
It can therefore be meaningfully plotted together with the original curves, when they are centered.

Apart from illustrating these properties, \cref{fig:4-means} provides two important validation checks for the estimation procedure proposed in \cref{sec:2,sec:3}.
Firstly, as stated, the estimated elastic full Procrustes mean is invariant to all Euclidean similarity transforms, as the mean shapes do not change with transformations of the input curves.
Secondly, when considering untransformed curves, the estimated elastic full Procrustes mean shapes are very compareable to elastic mean shapes estimated with the method proposed in \cite{Steyer2021}.
This is especially noteable when comparing the means in \cref{fig:4-means} (a) where the prominent \enquote{notch} in the center of the mean shape is similarly pronounced for the elastic and the elastic full Procrustes means, but not for the (non-elastic) full Procrustes mean.
Taken together this shows that the proposed mean estimation method provides elastic mean estimates in the setting of sparse and irregular curves, which are invariant to all shape-preserving transformations.


\section{Effect of the Penalty on the Estimated Mean}
\label{sec:4-penalty}
The elastic full Procrustes mean is given by the leading eigenfunction of the complex covariance surface, which was estimated using tensor product P-splines.
As a consequence, the order of the roughness penalty applied in the estimation of the covariance surface directly influences the shape of the estimated mean function.
This can be seen in \cref{fig:4-penalty}, where a smooth mean was estimated for different penalties.
\begin{figure}
  \centering
  \begin{subfigure}{\textwidth}
  \inputTikz{41-digit3-pen}
  \end{subfigure}
  \begin{subfigure}{\textwidth}
  \inputTikz{41-spiral-pen}
  \end{subfigure}
  \caption{Elastic full Procrustes mean under different penalties.
  Estimated using no penalty (left) and order 0/1/2 penalties (center-left/center-right/right) respectively, as well as 13 equidistant knots and linear B-splines on SRV level.
  Data: See \cref{fig:41-digit3-means-b,fig:41-spiral-means-b}}
  \label{fig:4-penalty}
\end{figure}
Here, in particular, the spiral mean shapes help to illustrate the penalty effect.
As mentioned in \cref{sec:4-means}, the spirals are evaluated in a way that makes them very sparse towards the end, leading to unstable mean estimates in that region.
By penalizing $p$-th order differences between neighbouring coefficients, the penalty helps to stabilize the covariance estimation (and thereby the mean) in regions where observations are sparse, where the shape of the function is consequently dominated by the penalty.

The mean function on the far left in \cref{fig:4-penalty} was estimated using no penalty.
We can see that the spiral shape is estimated well in the beginning (central part), where observations are dense, but becomes increasingly unstable and wriggly towards the end (outer part).
The mean function on the center-left was estimated using a zero order penalty.
A zero order penalty on a B-spline basis can be interpreted as a ridge-penalty on the basis coefficients, i.e.\ the basis coefficients (and therefore the estimated function) get shrinked towards zero in areas where observations are sparse.
It is important to note that the penalized covariance estimation is performed on SRV level and not on data curve level, which means the ridge penalty shrinks the estimated SRV mean towards zero.
However, this does not imply that the estimated mean on data curve level also gets shrunk towards zero.
In fact, a ridge penalty on SRV level only shrinks the \emph{absolute velocity} of the mean on data curve level towards zero, but should not directly influence the \emph{direction} of the mean curve.
This can be seen, when comparing the zero order penalty spiral mean (center-left) to the higher order penalty spiral means (right) and noting that, as a consequence of the decreased absolute velocity, it is relatively \enquote{shorter} towards the end when compared to the beginning of the spiral.
The higher order penalties may be interpreted as smoothing the SRV mean function towards a polynomal of degree $p-1$ in areas where the penalty dominates, which means a constant function for the first order penalty and a linear function for the second order penalty \parencite[see e.g.][435]{FahrmeierEtAl2013}.
\todo{Kann man das so einfach übertrage?}
Looking at the means for $p = 1,2$ in \cref{fig:4-penalty}, these differences are already hard to spot.
The order two mean (right) is slightly longer towards the end, indicating a speed up, which is caused by the penalization towards a global linearly (increasing) velocity, compared to a more conservative penalization towards a global constant velocity. 


\section{Elastic Full Procrustes Fits and Outliers}
\label{sec:4-pfits}
Although the elastic full Procrustes mean does not share rotation, scale and translation as the input curves, it is still possible to visually compare them, by plotting the mean together with the elastic full Procrustes fits.
In this thesis, the elastic full Procrustes fits are calculated on SRV level as $\widetilde q^\mathrm{EP} = (\omega^\mathrm{opt}\cdot \widetilde q \circ \gamma^\mathrm{opt} ) \sqrt{\dot\gamma^\mathrm{opt}}$, where the optimal rotation and scaling alignment $\omega^\mathrm{opt} = \lambda^\mathrm{opt} e^{\iu \theta^\mathrm{opt}}$ and optimal warping alignment $\gamma^\mathrm{opt}$ of each curve to the mean are taken from the last iteration of the mean estimation step.
In \cref{fig:4-pfits} the elastic full Procrustes mean and fits are plotted for the digits \enquote*{3} and simulated spiral datasets discussed in \cref{sec:4-means}, whith a random Euclidean similarity transform was applied to each curve.
\begin{figure}
  \centering
  \begin{subfigure}{0.48\textwidth}
    \inputTikz{4-digits3-pfits}
  \end{subfigure}
  \begin{subfigure}{0.48\textwidth}
    \inputTikz{4-spirals-pfits}
  \end{subfigure}
  \caption{Centered elastic full Procrustes fits (grey) and smooth means (red). Parameters: Second order penalty and linear B-splines, with 13 (left) and 19 (right) equidistant knots. Data: See \cref{fig:41-digit3-means-b,fig:41-spiral-means-b}}
  \label{fig:4-pfits}
\end{figure}
This alignment works very well for the simluated spirals and is more fuzzy for the digits, due to their greater variability in shape. 

In general, one has to be careful when interpreting these plots.
For example, it can be seen that the estimated mean does not necessarily follow the \enquote{center of mass} of the aligned curves. 
One reason for this is that the mean is calculated on SRV level, meaning we may set an arbitrary translation for the mean on data curve level.
Here, the Procrustes fits and the mean function are centered, but a natural alternative might be to have all curves start at the origin, with both choices leading to very different, but equally valid, visual representations.
It should also be taken into account that the mean is calculated from a smooth covariance surface, which was estimated using a roughness penalty (see \cref{sec:3-cov}), while the optimal rotation and scaling alignment was calculated by an approximation, where the observed normalized SRV curves are assumed piecewise-constant (see \cref{sec:3-pfits}).

%\begin{figure}
%  \centering
%  \begin{subfigure}{0.48\textwidth}
%    \inputTikz{4-digits3-pfits-srv}
%  \end{subfigure}
%  \begin{subfigure}{0.48\textwidth}
%    \inputTikz{4-spirals-pfits-srv}
%  \end{subfigure}
%  \caption{Elastic full Procrustes fits (grey) and piecewise-linear mean (red) on SRV level for digits (left) and spirals (right).
%  Parameters: See \cref{fig:4-pfits}.
%  Data: See \cref{fig:41-digit3-means-b,fig:41-spiral-means-b}}
%  \label{fig:4-pfits-srv}
%\end{figure}

%Außerdem sind die Procrustes Fits auch ein Punkt bei der MA, wo am ehesten etwas schieflaufen kann.
%Manche "Outlier Kurven" werden ja sehr schlecht dran-rotiert und dann auch recht klein. 
%As can be seen in \cref{fig:4-outliers}.
%Das kann natürlich Konsequenzen für das Warping-Alignment haben, dass sich dann nicht anpassen kann oder sogar schlechter wird.
%Es kann also passieren, dass diese Kurven nie gut dran rotiert und einfach klein werden \dots
%Bin mir noch nicht sicher, ob ich das als Bug oder Feature der Methode sehen soll.
%Die Means sehen am Ende ja trotzdem sehr gut aus (oder gerade deswegen? 
%Der full Procrustes Mean ist ja angeblich am "robustesten" unter den Procrustes Means). 
%Vielleicht gibt es für diese Kurven einfach kein gutes Alignment.
%
%Anhand des Kapitels kann man schön verdeutlichen, wo man wahrscheinlich noch etwas verbessern kann (z.B. mit dem von dir vorgeschlagenen Glätten der beobachteten Kurven mit Hilfe der geschätzten Kovarianz). 
%Aber ja, ab jetzt haben die Zungendaten auf jeden Fall Prio.

See Figure \cref{fig:4-outliers}. \dots \todo{Outlier diskutieren} See also \cref{app:a-warping}.
\begin{figure}
  \centering
  \begin{subfigure}{0.24\textwidth}
    \inputTikz{4-outlier-dist}
  \end{subfigure}
  \begin{subfigure}{0.24\textwidth}
    \inputTikz{4-outlier-bopt}
  \end{subfigure}
  \begin{subfigure}{0.48\textwidth}
    \centering
    \inputTikz{4-outlier}
  \end{subfigure}
  \caption{Left: Distribution of each curves elastic full Procrustes distance to the estimated mean.
  Center-left: Distribution of the scaling parameter of each elastic Procrustes fit onto the estimated mean.
  Right: Four largest outliers (grey) with respect to their distance to the estimated mean (red) for the \texttt{digits3.dat} dataset.
  Parameters: See \cref{fig:4-pfits}. 
  Data: See \cref{fig:41-digit3-means-b,fig:41-spiral-means-b}}
  \label{fig:4-outliers}
\end{figure}


\section{Variability in Tounge Shapes in a Phonetics Dataset}
\label{sec:4-tounges}
\textbf{[Paragraph: Einleitung, Ultraschallaufnahmen, Zungendaten]}

\textbf{[Paragraph: Datensatz, $(r,\theta)$ zu $(x_1, x_2)$, Arten der Varibilität $\rightarrow$ Wörter + Versuchspersonen + Wiederholungen]}

\textbf{[Paragraph: Fragestellung, Variabilität über Vokalkontext (quasi Wiederholung der Untersuchung aus dem Consulting Projekt)]}

\subsection*{Summary for Almond}
\begin{itemize}
  \item Want investiage how the effect of the flanking vowels \enquote*{aa} and \enquote*{ii} on tounge shape differs for the consonants 'd', 'l', 'n' and 's', while controlling for variability over six speakers and 5--7 repetitions of the same word per speaker.
  \item Analysis should be invariant with respect to size, rotation, translation and parametrization of the tounges.
    This can e.g.\ account for anatomical differences between speakers, or inaccuracies in the measurement process.
  \item \textbf{Idea:} Estimate a global mean function $q_0(t)$ from all curves and calculate the elastic full Procrustes residuals.
    They are given by the difference of elastic full Procrustes fits and the mean function on SRV level
    $$ \hat r_i = \left( \omega_i^\mathrm{opt} \cdot \widetilde q_i \circ \gamma_i^\mathrm{opt} \right) \sqrt{\dot\gamma_i^\mathrm{opt}} - \hat q_0 \,.$$
    We can perform analysis over the $\hat r_i(t)$. 
  \item A word is a combination of vowels and consonants $(C_v, C_c)$ with $C_v \in \{\mathrm{aa}, \mathrm{ii}\}$ and $C_c \in \{\mathrm{d},\mathrm{l},\mathrm{n},\mathrm{s}\}$, e.g.\ \enquote*{pada} corresponds to $(\mathrm{aa},\mathrm{d})$ as all words start with 'p'.
    We can index each word--speaker pair $\left( C_v, C_c, C_s \right)$ by $k$, where $C_s \in \{1,\dots,6\}$ and $k = 1, \dots, K$ with $K = 2 \cdot 4 \cdot 6 = 48$.
  \item Then we can gather the residuals $\hat r_{(k, rep_k)}(s_{(k,rep_k,j)})$ of the repetitions $rep_k$ (their number varies with $k$) of each word--speaker pair $k$, and estimate a \enquote*{mean} residual function for that word and speaker:
    $$ \hat r_k (t) = \hat r_{\left( C_v, C_c, C_s \right)} (t) \approx \hat \theta_k^\top b(t)$$
    Atm I use the mean basis here, which means linear B-Splines. Might be nicer to have something that is smoother.
  \item We are interested in the effect of the flanking vowels on tounge shape given, for each consonant-speaker pair, by this difference:
    $$ \hat d_{(C_c, C_s)}(t) = \hat r_{(\mathrm{aa}, C_c, C_s)} (t) - \hat r_{(\mathrm{ii}, C_c, C_s)} (t) = 
      \left( \hat\theta^\top_{(\mathrm{aa}, C_c, C_s)} - \hat\theta^\top_{(\mathrm{aa}, C_c, C_s)} \right) b(t)\,. $$
    Note that $\hat d(t) : [0,1] \rightarrow \mathbb{C}$ (two-dimensional)! The absolute differences over speakers and consonants are given by $\abs{\hat d_{(C_c, C_s)}(t)}$ are plotted in \cref{fig:4-diffs-vpn}.
\end{itemize}
\begin{figure}
  \centering
  \inputTikz{4-diffs-vpn}
  \caption{a caption}
  \label{fig:4-diffs-vpn}
\end{figure}
\begin{itemize}
  \item Finally we can average the differences over each speaker as
    $$ \hat d_{(C_c)} (t) = \left(\frac{1}{6} \sum_{C_s} \hat \theta^\top_{(C_c, C_s)} \right) b(t)$$
    or likewise as absolute difference  $\abs{\hat d_{(C_c)}(t)}$ plotted in \cref{fig:4-diffs}.
\end{itemize}
\begin{figure}
  \centering
  \inputTikz{4-diffs}
  \caption{Another caption}
  \label{fig:4-diffs}
\end{figure}
\begin{itemize}
  \item Man sieht, dass der Einfluss des flankierenden Vokals unabhängig vom Konsanten zu sein scheint.
    Das ist zumindest ein anderes Ergebniss, als in dem Consulting Projekt.
  \item Die Analysie ist hier auf SRV level.
  \item Sofern das alles so Sinn macht, könnte man jetzt noch Konfidenzintervalle undsowas dazu berechnen.
\end{itemize}
